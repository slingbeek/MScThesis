In our experiments of SAI the introduction of aerosols causes the signature stratospheric warming observed in previous studies \parencite{Ammann2010,bednarz2023climate}. Amospheric dynamical changes are predominantly governed by this thermal response, as they are in present day climate.

We observe the subtropical jet to shift equatorward and increase strongly under global warming, which is directly opposite of the equatorward and decreasing trend observed in previous studies \parencite{chenoli2017historical}. We also do not observe that the jet becomes wavier as was found by \textcite{martin2023}. It is not clear why our model exhibits this behaviour, but the dynamical change is consistent with the model's thermal response. We will thus treat this as the trend to be prevented by SAI. 

Within our model SAI prevents the changes in the STJ very well. Because the temperatures in the troposphere are kept generally at Reference levels, we do not observe large changes in the STJ. Most change appears to be contained within the upper regions of the jet where the stratosphere has warmed considerably, leading to a slight decrease in overall intensity and kinetic energy within the jet. 

The eddy-driven jet is situated in an area that experiences little to no thermal changes under SAI. The stark poleward shift and increase in strength we observe under high global warming, also observed in previous studies \parencite{Curtis_2020}, is prevented very well by SAI. As zonal wind speeds increase slightly and eddy activity decreases sligthly (mainly on the equatorward side), we conclude that the EDJ becomes sligthly less wavy under SAI. In this sense, SAI slightly overshoots its correction of the effects of high global warming. 

The most significant effect of SAI occurs in the upper stratosphere in the polar night jet, above the stratospheric warming. This change is undeniably caused by this thermal response as is clear from the integrated thermal wind. The PNJ shifts equatorward and becomes generally more broad, more wavy. This equatorward shift is of the same magnitude as observed under high global warming, but geographically distinct. It is not entirely clear if the effect of greenhouse gases have a compounding effect on the PNJ shift, but it is unlikely because of the regionally differing response. 

The PNJ has been shown to influence the EDJ, which could explain the changes in the EDJ. However, in the Southern Hemisphere this connection is strongest in late spring and summer \parencite{ceppi2019}. Notably the delayed breakdown of the PNJ favours a poleward shift of the EDJ. It is possible this effect is visible in Figure \ref{fig:EKE_U_zmdiff_DJF} in Appendix \ref{app_EDJ}, but the location of the maximum EKE in Figure \ref{fig:EDJ_maxloc} does not support this. 

In all scenarios the frequency of sudden stratospheric warming events appears to decrease. The atmosphere is less frequently in an SSW-like state, the yearly variability in 10 hPa temperature and zonal wind decreases significantly. This is expected under high global warming \parencite{jucker2021}, and SAI is not able to completely prevent this. It should be noted that one 30-year span for each scenario is not enough to confidently assess observed changes in the frequency of major SSWs. In the SH these events occur once every 22 years, thus a sample containing 30 years is not enough to determine a change in frequency. Additionally, this analysis only identifies if the atmosphere appears to be in a SSW-like state. For true assessment of SSWs, both major and minor events, daily model output is required.

Because we use only one run for each scenario, interpretation of these results remains limited. Our results do align with previous research, so this method of using an emulator is worth pursuing for the study of atmospheric dynamics in the context of SAI. 

The general difference between the gradual SAI and the rapid cooling SAI experiments is that the gradual SAI experiment has a slightly weaker response. It is not entirely clear why this response is weaker, whether the prolonged warming of the earth system has direct effects or if the slightly different aerosol burden is to blame. The 2-meter temperature in SAI 2080 in Figure \ref{fig:TREFHT_scens_ann}(d) shows a comparable response to SAI 2020. The warming hole is considerably larger in SAI 2080, as described in \textcite{pfluger2024} due to increased weakening of the atlantic meridional overturning circulation, which might explain the slight warming observed in the SH as well. It is also possible that the aerosol response is weaker because the warming hole creates a cool bias, leading to warming over the remainder of the globe. Nevertheless, the SAI 2080 is still an objectively good SAI scenario, able to prevent the effects of global warming, especially increasing surface temperature. 