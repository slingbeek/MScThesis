\begin{itemize}
    \item Increase of STJ over eastern Pacific could be signal from increased GHG's seen in Control.
    \item Changes in STJ and EDJ are significant, but pale in comparison to changes observed in Control. Lower stratosphere response is generally opposite that of Control, general weakening where there is great strengthening in Control.
    \item PNJ increases greatly, equatorward shift in all scenarios, possibly the great shift in SAI is a compounding of SAI and Control effect. Increase is only observed in SAI scenarios, about 8\% and 6\% increase of mean maximum wind speed.
    \item SSW's seem to decrease in all scenarios, mostly in Control but there is not really a significant difference between SAIs and Control. The earlier establishment in SAI is possibly a compound effect of SAI and increased GHG's as timing also shifts in Control, though less pronounced.
    \item SSW-like state of the stratosphere indicates lower frequency of SSW's, but it is only the SSW-like state based on monthly data. SSW's in end-beginning of two months will likely not show up in monthly data anyway, even more so if they are minor SSW's. 30 years of one run each is too few datapoints to make any definitive statements of how SSW's in CESM respond to SAI (and global warming) in the Southern Hemisphere. 
    \item Monthly data limits insights into SSW's, limitations with eddy-activity as well.
    \item Any interannual variability is not examined, only 20-year climatology.
    \item Coarse horizontal resulotuion limits evaluation of (sub)mesoscale atmospheric circulation patterns.
    \item Application of emulator to SAI 2080 scenario flawed, cooling from a different state than scenario emulator was built off leads to over- and undercooling.
    \item As in part I, using one run for each scenario limits interpretation of all results, not just SSW's.
\end{itemize}
