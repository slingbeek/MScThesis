\subsection{Climate change and geoengineering}
In the effort of limiting the effects global climate change, eliminating fossil fuels is the most important component. Regrettably, the complete elimination of fossil fuels in time to prevent the most disastrous effects of climate change and limit global warming to even 2°C is becoming increasingly unlikely (IPCC). With this prognosis, methods to temporarily lower the earth's global temperature are looked at to buy the global community time to lower atmospheric greenhouse gases. One such method is solar radiation management with stratospheric aerosol injections (SAI). 

Through injection of sulphate aerosols or their precursors at specific points in the stratosphere the earth's radiation budget is changed. At this high altitude the aerosols reflect short wave radiation, lowering the amount of sunlight reaching the earth's surface. In turn, the earth's surface temperature decreases. However, the long wave radiation emitted by the earth is absorbed by the aerosols too, resulting in warming in the stratosphere. This has effects on global atmospheric circulation patterns and subsequently on precipitation patterns. 

\subsection{Previous Research}
To investigate how the earth's atmosphere responds to the injection of sulphate aersols, a number of model studies has been performed. The most comprehensive project to date exploring this problem is the Stratospheric Aerosol Geoengineering Large Ensemble (GLENS) project (Tilmes et al. 2018). In this project the Community Earth System Model version 1 with the Whole Atmosphere Community Climate Model (CESM1(WACCM)) was used to perform simulations of a scenario that implemented SAI from 2020 onwards. The background used was the RCP8.5 emission scenario (IPCC/other paper). The model incorporated a feedback algorithm that adjusted the sulphur dioxide (SO$_2$) injection amounts at four injection points to maintain 2020 temperatures, or about 1.5°C  above pre-industrial levels. This included metrics related to the interhemispheric temperature gradient and the pole-to-pole temperature gradient. 

In 2020 similar simulations were done using the CESM2(WACCM6) with a variety of emission and reductions scenarios, including the scenario explored in the GLENS project (Tilmes et al. 2020). The CESM2(WACCM6) is a very large model with 70 vertical layers in the atmosphere and comprehensive chemistry in the troposphere up to the lower thermosphere and it incorporates a multimodal aerosol scheme (MAM4). This makes the model exceptionally well suited for analysis of SAI schemes, but does require a lot of computing time and resources. This limits its use for ensemble projects and exploration of various SAI schemes. 

\subsection{Evaluating the CAM6 Emulator}
One proposed solution is to use the aerosol field resulting from the CESM2(WACCM6) simulations and prescribing this field to a smaller, less cumbersome model. This was done in Pflüger et al. (2024, pre-print) with the Community Atmosphere Model version 6 (CAM6). The aerosol field was adjusted using a feedforward-feedback algorithm only adjusting for GMST. The fist part of this thesis is the validation of this method and model in its use in simulating SAI scenarios based on the more comprehensive CESM2(WACCM6) results. 
We formulate the following research questions for the first part:

\begin{enumerate}
    \item Does the CESM2(CAM6) succeed in maintaining global surface temperature through SAI, including spatial patterns?
    \item How does the CESM2(CAM6) SAI simulation perform  compared to the CESM2(WACCM6) SAI simulation it was based on?
\end{enumerate}

\subsection{Southern Hemisphere Atmosphere}
In the second part of this thesis the CESM2(CAM6) simulation of the SAI scenario is used to investigate the effects of SAI on atmospheric dynamics in the Southern Hemisphere high latitudes. Additionally, a new scenario of rapid late-century cooling through SAI is introduced. Comparison between these scenarios can provide insight in the memory of the earth system and efficacy of rapid cooling compared to gradual cooling using SAI. 

The Southern Hemisphere high latitudes are of special interest in the context of preventing climate catastrophe as large scale atmospheric dynamical changes affect the Antarctic ice sheet. Changes in temperature, precipitation and wind fields affect the surface mass balance of the ice sheet. Indirectly these changes affect the ice sheet through changes in the ocean, mainly in the rate and location of overturning circulation. 

We formulate the following research quesitons for the second part:

\begin{enumerate}
    \item How does SAI affect the Southern Hemisphere mid-level atmospheric circulation?
    \item How does SAI affect the Southern Hemisphere high latitude atmospheric circulation at high altitude (Polar Night Jet)?
    \item How does the late-century SAI scenario differ from the gradual SAI scenario?
\end{enumerate}

\textcolor{teal}{ik zou heeeel graag wat feedback willen op m'n onderzoeksvragen, ik vind het goed formuleren van waar ik nou naar heb gekeken nog best lastig blijkt}