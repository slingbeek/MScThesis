\documentclass[12pt]{article}
\usepackage[margin=2.5cm]{geometry}
\usepackage{amsmath}
\usepackage{amssymb}
\usepackage{mathtools}  
\usepackage{diffcoeff}  
\usepackage{siunitx}
\usepackage[table,xcdraw]{xcolor}
\usepackage{amsfonts}
\usepackage{subfiles}
\usepackage{hyperref}
\usepackage{newtxtext, newtxmath}
\usepackage{enumitem}
\usepackage{titling}
\usepackage{listings}
\usepackage{float}
\usepackage{hyperref}
\setlength{\droptitle}{-6em}
\usepackage{placeins}
\usepackage{graphicx}
\usepackage{subcaption}
\usepackage[utf8]{inputenc}
\usepackage{fancyhdr}
\usepackage{color}
\usepackage{xspace}
\usepackage{pifont}
\usepackage{tikz}

%%% This is about changing the headers and footers (i.e. Top and bottom of the page)
\pagestyle{fancy}% use fancyheaders with the bar on the top
\fancyhf{} % Clear the normal style
\fancyhead[L]{\bfseries\leftmark} %this places the section number and name in the top left
\fancyhead[R]{\bfseries\thepage}% this places the pagenumber in the top right
%%%% Add the bibliography with some settings:
% package:
% \usepackage[square, comma, numbers, sort&compress ]{natbib}
\usepackage[style=authoryear,citestyle=authoryear,maxcitenames=2,maxbibnames=20,uniquename=false,uniquelist=false,eprint=false]{biblatex}
\addbibresource{bibbestand.bib}

\renewbibmacro*{doi+eprint+url}{%   
  \iftoggle{bbx:url}     
    {\iffieldundef{doi}{\usebibmacro{url+urldate}}{}}     
    {}%   
  \newunit\newblock   
  \iftoggle{bbx:eprint}     
    {\usebibmacro{eprint}}     
    {}%   
  \newunit\newblock  
  \iftoggle{bbx:doi}     
    {\printfield{doi}}     
    {}}

%%%%% frontmatter/mainmatter/backmatter:
\newcommand\frontmatter{%
    \cleardoublepage
    \pagenumbering{roman}} %small Roman numbers

\newcommand\mainmatter{%
    \cleardoublepage
    \pagenumbering{arabic}} %normal numbers

\newcommand\backmatter{%
    \cleardoublepage %% double page style
    %\clearpage %% single page style
    \pagenumbering{Roman}} %capital Roman numbers


\begin{document}
\newgeometry{margin=1.5cm} %% Special margins on the titlepage. It is also possible to set each margin separately; see package Geometry.
%%% Titelpagina in een apart bestand

\begin{titlepage} %%% This is your titlepage. Everything should match the conditions as they are right now for a physics thesis; hopefully the conditions won't change soon. In order to change this page into your very own title page, replace the noted parts with your own (like 'Your title').
	\noindent
	% \begin{minipage}{0.4\textwidth}
	% 	\includegraphics[width=\linewidth]{uulogo} %%%% Logo of the UU. English version is also available
	% \end{minipage}

    \par\vspace{1cm}
    
    \begin{flushright}
    {\LARGE\bfseries Utrecht University \par} %Enter your faculty here. It is probably right already (usually in Dutch).
    \end{flushright}
	\vspace{1cm}
    \begin{center}
    {\huge\bfseries Emulating SAI Scenarios in CESM2 and the Effects on the High-Latitude Southern Hemisphere Atmospheric Circulation\par} %% ENTER TITLE HERE
    \end{center}
	\vspace{1cm}
    {\scshape\Large Master Thesis\par}
	\vspace{0.5cm} % You change this to set the distance 'Bachelor Thesis' to 'Your name', and all the other vspaces to set the heights of everything.
	{\Large\itshape Simone Lingbeek\par} %% ENTER NAME HERE
    \vspace{0.5cm}
    % \centering
    % \fbox{ %% fbox puts a black border around your picture
    % \includegraphics[height=0.35\textheight]{Clipboard01} %% If you use a picture, place it here. If you don't, delete this, from '\centering' to '\par'.
    % %% change the height ration until it fits.
    % } %% the fbox ends here
    % \vspace{0.5cm}
    % \par
    \raggedleft
	{\Large\itshape Supervisors}:\par\vspace{0.25cm}
	{\large Dr. Claudia Wieners \textsc{Supervisor}\par} %% first supervisor
    Institute for Marine and Atmospheric research Utrecht\par %% their institute
    \vspace{0.25cm}
    {\large Dr. Michiel Baatsen \textsc{Supervisor}\par} %%second supervisor
    Institute for Marine and Atmospheric research Utrecht\par %% their institute
    \vspace{0.25cm}
    {\large Jasper de Jong \textsc{Daily supervisor}\par} %%daily supervisor
    Institute for Marine and Atmospheric research Utrecht %% their institute
    %% copy/paste from \vspace{0.25cm} to here for more supervisors.

	\vfill
	{\large \today\par}%%% The date. Replace \today with the necessary date if your necessary date isn't today.
    
\end{titlepage}    

\restoregeometry %%% Restores the margins for the rest of the document.

\frontmatter

\begin{abstract}
  \noindent The implementation of adequate and timely policies to prevent the further increase of greenhouse gases in our atmosphere and the most disastrous effects of global warming is becoming increasingly unlikely. In this context, stratospheric aerosol injections (SAI) could provide a solution by (temporarily) decreasing global mean surface temperature. The study of SAI as a climate intervention requires earth system models capable of resolving comprehensive atmospheric chemistry and dynamics. Here we validate a method that makes use of the results of the comprehensive atmospheric model CESM2(WACCM6) to simulate SAI with the simpler CESM2(CAM6). We show that this method is succesful in replicating the experiment, reproducing surface temperature and precipitation trends within the range of model-variability. The atmospheric thermodynamical changes caused by SAI are replicated in our model as well, especially for the Southern Hemisphere. We then use our model to conduct experiments with two SAI scenarios - a gradual SAI starting in 2020 and a rapid cooling SAI scenario starting in 2080, both with SSP5-8.5 as background. We study the effects of SAI on the large-scale atmospheric circulation of the Southern Hemisphere. We find that both SAI scenarios are able to prevent the changes in the lower stratosphere observed under SSP5-8.5. In the upper stratosphere SAI leads to a much stronger polar night jet. SAI is also not able to prevent the strong decrease in the frequency of sudden stratospheric warming events as observed under SSP5-8.5. We find a slightly weaker response to SAI in the rapid cooling SAI scenario, but overall trends are identical to the gradual SAI scenario. 
\end{abstract}
\newpage

\tableofcontents

\newpage
\section*{List of Acronyms}
\begin{description}[align=right, labelwidth=2cm, labelsep=1cm]
  \item[\textbf{SRM}] Solar radiation management
  \item[\textbf{GMST}] Global mean surface temperature
  \item[\textbf{SAI}] Stratospheric aerosol injections
  \item[\textbf{STJ}] Subtropical jet
  \item[\textbf{EDJ}] Eddy-driven jet
  \item[\textbf{PNJ}] Polar night jet
  \item[\textbf{SSW}] Sudden stratospheric warming event 
  \item[\textbf{KE}] Kinetic energy
  \item[\textbf{EKE}] Eddy kinetic energy     
\end{description}
\newpage

\mainmatter

\section{Introduction}
\subsection{Climate change and geoengineering}
In the effort of limiting the effects global climate change, eliminating fossil fuels is the most important step to take. Regrettably, the complete elimination of fossil fuels in time to prevent the most disastrous effects of climate change and limit global warming to even 2°C is becoming increasingly unlikely. Even with all currently committed climate action goals, projections show the earth warming significantly above the 1.5°C and 2°C targets from the Paris Agreement \parencite{NDCsynth}. With this outlook, methods to temporarily lower the earth's global temperature are looked at to buy the global community time to lower atmospheric greenhouse gases. One such method is solar radiation management with stratospheric aerosol injections (SAI). 

Geoengineering can be seen as a toolbox of methods that change the earth's climate system to achieve a desired effect. Limiting global mean surface temperature (GMST) increase is the primary goal of geoengineering. The methods of geoengineering can be divided into two basic categories, Carbon Dioxide Removal (CDR) and Solar Radiation Management (SRM) \parencite{shepherd2009}. CDR focuses on lowering the amount of greenhouse gases in the atmosphere, by capturing CO$_2$ directly or enhancing and facilitating natural processes to speed up the extraction of CO$_2$ from the atmosphere or oceans. SRM on the other hand focuses on altering the earth's radiation budget. This can be done by increasing the amount of long wave radiation the earth emits into space, for instance with cirrus cloud thinning, but the most widely discussed approach is increasing how much short wave radiation from the sun is reflected back into space, i.e. increasing the planetary albedo. This is done for instance by making deserts more reflective or making clouds brighter like through marine cloud brightening \parencite{reflecting}. 

\subsection{Geoengineering in the Form of Stratospheric Aerosol Injections}
This thesis considers SRM in the form of stratospheric aerosol injections (SAI). Through injection of sulphate aerosols or their precursors at specific points in the stratosphere the earth's radiation budget is changed. At this high altitude the aerosols reflect short wave radiation, lowering the amount of sunlight reaching the earth's surface and subsequently GMST is lowered. It has been argued that SAI is one of the most feasible options for SRM \parencite{lenton2009,shepherd2009}.

A direct result of SAI is the absorption of long wave radiation emitted by the earth by the aerosols, resulting in warming in the stratosphere \parencite{Ammann2010}. This impacts the global atmospheric circulation patterns and precipitation patterns. It is also important to keep in mind that the effects other than warming caused by increased greenhouse gases are still present in the earth system, for instance ocean acidification will continue if CO$_2$ concentrations continue to rise. The introduction of sulphate aerosols has unintended consequences too, including in delayed repair of the ozone hole and an increase in acid deposition, and interference with cloud formation. 

Eventhough complete understanding of the physical consequences of SAI is not within reach as of now, the physical feasibility in regards to limiting global warming is rather certain. Practically, the succesful deployment of SAI is only possible if the global community can reach consensus on its employment and can ensure no party deploys SAI undemocratically and to the detriment of others. Even then, long-term political and economical stability are crucial, as an abrupt halt would lead to rapid warming \parencite{robock2009}. 

The employment of SAI is not as straight-forward as appears at first glance, and before any decisions can be made about SRM, with SAI or through other means, more research is needed. The National Academy of Sciences made extensive recommendations for solar geoengineering research \parencite{reflecting}, including research into the impacts on atmospheric processes, the climate response, desigining a monitoring system, how to govern solar geoengineering activities and ethical considerations for current and future generations. 

\subsection{General Structure}
In this thesis we investigate the climate response to SAI, more specifically the atmospheric dynamics of the Southern Hemisphere. We will use the results from a previous study on SAI conducted with an earth system model with an extensive atmospheric component and comprehensive atmospheric chemistry to build an emulator with a model with a smaller atmospheric component that incorporates no atmospheric chemistry. We do this primarily to save on computation time. We will validate this experiment in Part I, and use it to assess the impact of SAI on the Southern Hemisphere atmospheric circulation in Part II. We will focus on changes in the large scale components of the atmospheric circulation. 

The southern hemisphere is of particular interest because of the Antarctic Ice Sheet that is home to a number of tipping points. The triggering of these tipping points can potentially lead to disastrous levels of sea level rise and once these tipping points are reached, there is no way to reverse the effects. Understanding the dynamics of the Southern Hemisphere and the response to SAI is crucial to accurately predict the future evolution of these tipping points and if triggering them can be avoided with the help of SAI. 

\newpage
\section{Introduction Part I}
\subsection{Previous Research on SAI in Earth System Models}
Large scale studies on SRM started with the Geoengineering Model Intercomparison Project (GeoMIP) \parencite{geomip2011}. Experiments reduced the incoming solar radiation either directly or through injection of SO$_2$ in the stratosphere at one point on the equator, which was then distributed in the atmosphere very quickly by the zonal circulation and on larger time scales the meridional circulation.

It was found that injection of aerosols at just the equator, and uniform solar radiation reduction lead to over-cooling of the tropics and under-cooling of the poles. \textcite{kravitz2016} proposed a different approach, where climate targets were considered the main goal and how to reach those goals through SAI was viewed as a design problem. One set of targets proposed was the global mean surface temperature ($T_0$), the interhemispheric temperature gradient ($T_1$) and the pole-to-equator temperature gradient ($T_2$). The proposed SAI strategy included four injection points, two on each hemisphere, and a feedback algorithm that adjusts the injection rate at those points to achieve the temperature goals. This method was applied by \textcite{kravitz2017} and \textcite{macmartin2017}, and then in the Geoengineering Large Ensemble (GLENS) project \parencite{tilmes2018}. 

\subsection{The GLENS Project and Subsequent CESM2 Simulations}
The GLENS project is a 20-member ensemble of gradual SAI simulations. From 2020 onwards SO$_2$ is injected at four injection points at $\pm$ 15°N and $\pm$ 30°N about 5 km above the tropopause. A feedback-control algorithm is used to adjust the injection amounts at each point individually based on the departure of the $T_{0,1,2}$ temperatures defined by \textcite{kravitz2016} from 2020 levels.

The GLENS project was performed using the Community Earth System Model version 1 (CESM1) \parencite{hurrell2013}, with the Whole Atmosphere Community Climate Model (WACCM) as its atmosphere component. This model uses a 0.9° latitude $\times$ 1.25° longitude rectangular grid with 70 vertical layers that reach up to 140 km, or about 10$^{-6}$ hPa. It includes comprehensive atmospheric chemistry for the middle atmosphere, incorporating ozone chemistry and chemistry relating to stratospheric sulfate formation. A simpler chemistry scheme is used for the troposphere. Aerosol chemistry is coupled to WACCM through the three-mode version of the Modal Aerosol Module (MAM3).

After the GLENS project the method using four injection points and a feedback algorithm was further explored using the more recent Community Earth System Model version 2, also using a newer version of the WACCM, now version 6 (CESM2(WACCM6)) \parencite{tilmes2020}. WACCM6 has the same horizontal and vertical resolution, but now includes comprehensive chemistry from the troposphere up to the lower thermosphere. The MAM4 modal aerosol scheme is used for the troposphere and stratosphere. We will refer to this model as `WACCM'.

Because WACCM is comprehensive in both vertical resolution and atmospheric chemistry, it is well-suited to simulate SAI scenarios. However, due to this it is also a very cumbersome model, requiring more computing time and resources compared to less comprehensive models. This limits its use in ensemble studies and studies considering a large range of scenarios. Computational cost also limits its use for simulations with a longer timeseries at higher resolution, required to study smaller scale phenomena like tropical cyclones. 

\subsection{Building an Emulator for SAI with CESM2(CAM6)}
Reducing the amount of computing resources needed for an experiment can be done by using a smaller, less comprehensive model and using previous results from WACCM as external forcing to the model, essentially building an emulator. \textcite{pfluger2024} introduce a method for building such an emulator of WACCM to model SAI. They use the same CESM2 model configuration is used as in \textcite{tilmes2020}, but the Community Atmosphere Model version 6 (CAM6) is used for the atmosphere component \parencite{danabasoglu2020}. This model has only 32 vertical levels and no atmospheric chemistry. We will refer to this model as `CAM'. The aerosol field established in the WACCM simulation is used as an external forcing. To maintain 2020 levels of GMST, the necessary globally averaged aerosol optical depth (AOD) field is found though the same feedforward-feedback algorithm. All other forcing fields are then scaled accordingly. In contrast to the WACCM simulations, the temperature gradients $T_1$ and $T_2$ are not adjusted for. 

The WACCM SAI experiment is well-suited to build an emulator for, as the employment of SAI is used as a means to reach a climatic goal. It prescribes an aerosol burden in response to model behaviour, in stead of gauging the model response to a certain aerosol burden. This removes the amount of SO$_2$ injected into the stratosphere from the experiment design and allows for translation of the aerosol distribution to another model that might respond differently to the introduction of aerosols.

\subsection{Scenarios Part I}
There are two simulations performed with CAM and WACCM. The first simulation follows the historical spin-up and is continued by the SSP5-8.5 scenario \parencite{RIAHI2007887}. The second simulation branches from the first simulation in 2020 and from then on introduces SAI to stabilise temperatures using the SSP5-8.5 scenario as background. Here we will refer to it as the gradual SAI scenario. A schematic view of the GMST response in each scenario is shown in Figure \ref{fig:schematic_scens_pt1}.

\begin{figure}[H]
    \centering
    \includegraphics[width=0.6\linewidth]{images/schematic_scens_pt1.png}
    \caption{Schematic of the global mean surface temperature in the two scenarios used in Part I, the SSP5-8.5 and gradual SAI scenarios.}    
    \label{fig:schematic_scens_pt1}
\end{figure}


\newpage
\subsection{Research Questions Part I}
The fist part of this thesis is the validation of this method and model in its use as an emulator, simulating SAI scenarios based on the WACCM results. Two experiments, a high-warming scenario and a gradual SAI scenario, are used to assess the impact of SAI in each model (CAM and WACCM). We calculate the three temperature targets for all models and we will look at the model results for 2-meter temperature for additional insight into the performance of CAM in regards to regulating surface temperatures. We discuss both annual and seasonal patterns. Additionally, we discuss precipitation patterns.

The most important difference in dynamics between the two models lies in the inclusion of ozone chemistry and thus the interaction of stratospheric aerosols and ozone. It is known that sulphate aerosols in the stratosphere accelerate chemical ozone loss through halogen activation and strengthening of the polar vortex \parencite{bednarz2023ozone}. CAM does not include this process, and assessing whether the dynamical response to SAI is comparable to WACCM is thus important. To this end we assess any differences in the vertical profile between the two models, we compare the zonally averaged potential temperature and zonal winds. 

We formulate the following research questions for Part I:
\newline

\noindent \textit{When applying the stratospheric aerosol emulator to a gradual SAI scenario in CESM2(CAM6), are we able to reproduce from \textcite{tilmes2020}
    \begin{enumerate}[label=\roman*]
        \item the temperature targets $T_0$, $T_1$ and $T_2$?
        \item the spatial and seasonal variations of temperature, precipitation and general aspects of the atmospheric circulation?
    \end{enumerate}
}

\subsection{Results SAI in WACCM}
The experiments performed with WACCM \parencite{tilmes2020} showed that surface temperatures were generally controlled with SAI. Under SAI the tropics warmed slightly, the mid-latitudes cooled slightly and the poles warmed slightly as well. Zonal mean precipitation showed an increase in the tropics and a decrease or no significant change everywhere else. An apparent warming hole over the Northern Atlantic was observed in all simulations and was likely related to observed changes in the Atlantic Meridional Overturning Circulation. No changes in seasonal patterns were discussed, nor were changes in potential temperature and atmospheric circulation. 


\newpage
\section{Introduction Part II}
\subsection{Rapid Cooling with SAI as an Emeregency Intervention}
As sufficient climate policies to prevent global warming of 1.5°C or even 2°C \parencite{NDCsynth}, it is unlikely the global community will implement a proactive gradual SAI scenario in time as well. Employing SAI much later on after prolonged heating of the climate system is a realistic, more reactive, scenario. The earth could be cooled very rapidly, allowing the end-of-century GMST goals to be reached after the climate system has endured an even longer period of warming than it has to date. The effects of such an intervention are largely uncertain. While it is rather certain that SAI could lower GMST, it is not certain what effects of previous warming can be reversed, if at all. The SAI 2080 scenario introduced in \textcite{pfluger2024} is such a scenario, SAI is employed from 2080 onwards to achieve rapid cooling to 2020 levels. 

\subsection{Scenarios Part II}
For the second part of this thesis we consider the two CAM simulations from Part I, the SSP5-8.5 and the gradual SAI simulations, and the SAI 2080 scenario from \textcite{pfluger2024}. We will refer to this simulation as the rapid cooling SAI simulation. The rapid cooling simulation is branched from the SSP5-8.5 simulation in 2080, SAI is then deployed to restore $T_0$ to 2020 levels. The control algorithm is adjusted up to the first six years to prevent extremely high aerosol concentrations that would result in too rapid cooling. A schematic view of the GMST response for this scenario is shown in Figure \ref{fig:schematic_scens}.

\begin{figure}[H]
    \centering
    \includegraphics[width=0.65\linewidth]{images/schematic_scens.png}
    \caption{Schematic of the global mean surface temperature in the two scenarios used in Part I, the SSP5-8.5, the gradual SAI and the rapid cooling SAI scenarios.}  
    \label{fig:schematic_scens}  
\end{figure}


\subsection{The Southern Hemisphere Atmopshere}
The atmosphere is home to a number of large scale circulation phenomena. Though the two hemispheres are symmetrical in the occurence of these phenomena, the exact location, frequency of occurence and other behaviour of these phenomena differs between the two hemispheres. The Southern Hemisphere circulation is generally more stable than its counterpart in the Northern Hemisphere. In this thesis we focus on three so-called jets, large scale circulation phenomena that persist throughout the year or in a specific season.
We discuss the subtropical jet and the the eddy-driven jet in the lower stratosphere (up to 50 hPa), and the polar night jet in the upper stratosphere (above 50 hPa).


\subsubsection{The Subtropical Jet}
The subtropical jet (STJ) forms at the convergence of the upper branches of the Hadley and Ferrel cells in the lower stratosphere, with a maximum around 200 hPa, see Figure \ref{fig:jetcrosssection} (the Southern Hemisphere is analogous to the Northern Hemisphere). Influenced by the Coriolis force westerly winds form here, the magnitude of which is determined by the temperature gradient below the convergence zone \parencite{zolotov2018variability}. The STJ occurs year-round but is strongest in winter, in tandem with the stronger winter Hadley cell, where it is located around 30°S.

In the past decades, the southern hemisphere STJ has been observed to shift poleward and slightl decrease in wind speed, though not significant \parencite{zolotov2018variability}. The STJ was also observed to be increasingly wavy \parencite{martin2023}. Under high global warming the STJ is projected to shift further poleward and increase in strength \parencite{chenoli2017historical}. 

Simulations with SAI at singular injection points showed a small but significant decrease in the strength of the STJ \parencite{richter2017}. From historical simulations it has been argued as well that anthropogenic aerosols contribute to deceleration \parencite{wang2020}.

\begin{figure}[t]
    \centering
    \includegraphics[width=0.8\linewidth]{images/Jetcrosssection.png}
    \caption{Cross section of the Northern Hemisphere atmospheric ciruclation, with the suptropical jet and polar jet (or eddy-driven jet) in relation to the circulation cells. Image by: Original: National Weather Service, JetStream Vector: Sleske - Own work based on: Jetcrosssection.jpg, CC BY-SA 4.0, \href{https://commons.wikimedia.org/w/index.php?curid=75169357}{https://commons.wikimedia.org/w/index.php?curid=75169357}}  
    \label{fig:jetcrosssection}  
\end{figure}


\subsubsection{The Eddy-driven Jet}
At the divergence of the upper branches of the Ferrel and Polar cells another jet forms, the polar jet in Figure \ref{fig:jetcrosssection}. This jet is at a lower altitude and stronger than the STJ. However, it is also much wavier than the STJ \parencite{martin2023}. Because of this, and to avoid confusion with the polar night jet, we will refer to this jet as the eddy-driven jet (EDJ). 

The EDJ has been observed to become increasingly wavy in the past decades, also shifting poleward, even more strongly than the STJ, but no significant trend in the wind speed was observed \parencite{martin2023}. Under high global warming the EDJ is projected to shift further poleward \parencite{Curtis_2020}.

It is not certain how the EDJ will respond under SAI, though it has been shown that the EDJ is influenced by the strength of the polar night jet in the upper stratosphere, where a strengthened jet leads to a poleward shift of the EDJ \parencite{kidston2015stratospheric}. Any changes in the polar night jet due to SAI are thus likely to cause changes in the EDJ as well.

\subsubsection{The Polar Night Jet}
The polar night jet (PNJ) forms in the upper stratosphere during winter in response to the large equator-to-pole meridional temperature gradient. The pole becomes encircled with a belt of strong westerly winds in the upper parts of the stratosphere, above $\approx$50 hPa at around 60°S \parencite{lee2021}. 

With high warming the PNJ is projected to increase in strength, most notably in late SH spring, effectively delaying the ultimate breakdown of the PNJ \parencite{ceppi2019}.

The warming in the lower stratosphere due to SAI leads to an increase of the equator-to-pole meridional temperature gradient, causing increasing zonal winds via the thermal wind balance. This effect is strongest in the late winter/early spring \parencite{bednarz2023injection,bednarz2023ozone}.

When disturbances near the surface propagate to the stratosphere, the PNJ can suddenly weaken, leading to a sudden increase in temperatures of the polar stratosphere and sometimes a complete breakdown of the jet. These events are called sudden stratospheric warming events (SSWs) and occur very rarely in the Southern Hemisphere in the current climate. Their frequency is projected to strongly decrease with increasing global warming \parencite{jucker2021}. 


\subsection{Antarctica and the Southern Hemisphere Atmosphere}
The high-latitude Southern Hemisphere is of particular interest in the context of climate change and the prevention of it. The Antarctic Ice Sheet (AIS) could contribute greatly to global sea level rise if it were to become unstable under global warming. Observed instabilities of the West-Antarctic Ice Sheet (WAIS) alone could contribute to significant sea level rise \parencite{IPCC_2021_WGI_Ch_9}. 

Large scale atmospheric dynamical changes affect the Antarctic ice sheet, changes in temperature, precipitation and wind fields alter the surface mass balance of the ice sheet. More distantly dynamical changes affect the ice sheet through changes in the ocean, mainly in the rate and location of overturning circulations. A warmer ocean together with a warmer atmosphere can lead to the loss of ice shelves \parencite{WANG2023} and their disappearance has been observed to increase the flow of glaciers feeding into the ocean \parencite{scambos2004}, contributing to mass loss of the AIS. 

The Southern Hemisphere high stratosphere has low variability in the current climate, but any changes could have far-reaching effects on the surface climate. It has been observed that SH stratospheric polar vortex weakekening contributes to climate anomalies in Australia and New Zealand, southeast Africa and southern South America. Additionally, the wind stress over the ocean around Antarctica is weakened and the Ross and Amundsen seas experience warmer climate \parencite{domeisen2020}. There is thus a clear interaction between the atmospheric dynamics and local climate in the Southern Hemisphere. 

The effect of SAI on the Antarctic ice sheet has been studied by \textcite{mccusker2015}, who found that a rapid introduction of sulphate aerosols in the stratosphere could not prevent the collapse of the WAIS. \textcite{sutter2023} found similar results. However, both studies use very simple aerosol schemes, that for instance only inject aerosols in the tropics. These types of schemes are known to lead to over-cooling of the tropics and under-cooling of the poles. As stated before, the studies using a feedback-control algorithm to maintain the $T_{0,1,2}$ temperature targets were laregely successful in this regard. 


\subsection{Research Questions Part II}
In the second part of this thesis the effect of SAI on the high-latitude Southern Hemisphere atmospheric dynamics is investigated, both in the gradual and rapid cooling scenarios. The focus here lies on large scale circulation patterns, as they largely dictate local climate in the Southern Hemisphere. 

We formulate the following research quesitons for Part II:
\newline

\noindent \textit{What are the impacts of the gradual SAI scenario on the Southern Hemisphere
    \begin{enumerate}[label=\roman*]
        \item subtropical and eddy-driven jets in the lower stratosphere?
        \item polar night jet and sudden stratospheric warming events in the upper stratosphere?
    \end{enumerate}
How do the results for i and ii change under the rapid cooling SAI scenario?
}
\newpage

\part{Model Validation}

\section{Methods Part I}
\subsection{Building the Emulator for SAI simulations}
The emulator for SAI simulations is introduced in \textcite{pfluger2024}, it implements SAI via prescribed aerosol fields, as opposed to sulphate injections that result in aerosol fields through model physics. As per \textcite{pfluger2024}, the protocol works as follows:
\begin{itemize}
    \item Every year, observe the deviation of GMST from the target.
    \item Based on past GMST deviations, infer the level of SAI - expressed in terms of global mean aerosol optical depth (AOD) - which is necessary to achieve the desired target.
    \item Use the AOD to scale all SAI-related aerosol fields appropriately.
    \item Feed the scaled fields into CAM6.
\end{itemize}

The first two steps are implemented via the feedforward-feedback control algorithm as established in \textcite{kravitz2017}. The control algorithm stabilises only GMST, not inter-hemispheric and equator-to-pole temperature gradients like the simulation performed by \textcite{tilmes2020}.

The prescribed aerosol fields are the averaged aerosol fields from the WACCM simulation, this simulation is called the \textit{Geo SSP5-8.5 1.5 scenario} in \textcite{tilmes2020}. The fields are normalised, averaged and fit, to then arrive at an amplitude for each aersol component.

\textcolor{teal}{Stukje over aerosolveld hoe het eruit ziet + jaarlijkse totale massa figuren (?)}

\subsection{Definition of Scenarios and Time Periods for Part I}
There are two simulations performed with each CESM2 configuration. The first simulation follows the historical spin-up and is continued by the SSP5-8.5 scenario. The second simulation branches from the first simulation in 2020 and from then on introduces SAI to stabilise temperatures using the SSP5-8.5 scenario as background. Here we will refer to it as the gradual SAI scenario. 

Throughout this first part, three time periods are used to visualise and interpret the results from the simulations. For each period the 20-year mean is taken, unless specified otherwise. These periods are defined as follows:

\begin{itemize}
    \item \textbf{Reference} The period 2016-2035 of the SSP5-8.5 simulation.
    \item \textbf{Control} The period 2080-2099 of the SSP5-8.5 simulation.
    \item \textbf{SAI 2020} The period 2080-2099 of the gradual SAI simulation.
\end{itemize} 
\textcolor{teal}{misschien beter in tabelvorm}

\subsection{Vertical Interpolation}
The atmospheric vertical levels of CESM are defined as hybrid levels. However, for ease of computation and visualisation, conversion to pressure levels is preferred. To this end, a logarithmic interpolation scheme is used, that takes the model output at hybrid levels and projects them onto the appropriate pressure levels. For CAM the model output is converted to 34 pressure levels ranging from 3.5 to 993 hPa. For WACCM the SSP5-8.5 model output was pubplished in pressure levels, 19 pressure levels ranging from 1 to 1000 hPa. The WACCM gradual SAI scenario model output is converted to those same 19 pressure levels. \textcolor{teal}{moet dit uitgebreider?}

No interpolation is needed of the horizontal grids as these are identical in all models. 

\subsection{Temperature Targets}
The WACCM simulations use the feedback algorithm to maintain three temperature targets in their SAI scenario. These temperature targets are defined in \textcite{kravitz2016} as the projection of $T(\psi)$ onto the first three Legendre polynomial functions of $\sin(\psi)$, resulting in

\begin{equation}\label{eq:TdA}
    \begin{split}
        T_0 &= \frac{1}{A} \int\displaylimits_{-\pi/2}^{\pi/2} T(\psi) \mathop{dA},\\
        T_1 &= \frac{1}{A} \int\displaylimits_{-\pi/2}^{\pi/2} T(\psi) \sin(\psi) \mathop{dA,}\\
        T_2 &= \frac{1}{A} \int\displaylimits_{-\pi/2}^{\pi/2} T(\psi) \frac{1}{2}(2\sin^2(\psi) -1) \mathop{dA},
    \end{split}
\end{equation}

\noindent where $\psi$ is latitude, $T(\psi)$ is the zonal-mean temperature for each latitude, and A the area-weighted latitude, defined as

\begin{equation}\label{eq:dA}
    \mathop{dA} = \cos(\psi) \mathop{d\psi} \Rightarrow A = \int\displaylimits_{-\pi/2}^{\pi/2} \cos(\psi) \mathop{d \psi} = 2.
\end{equation}

\noindent Combining Eqs. \ref{eq:TdA} and \ref{eq:dA} we find 

\begin{equation}\label{eq:Tpsi}
    \begin{split}
        T_0 &= \frac{1}{2} \int\displaylimits_{-\pi/2}^{\pi/2} T(\psi) \cos(\psi) \mathop{d\psi},\\
        T_1 &= \frac{1}{2} \int\displaylimits_{-\pi/2}^{\pi/2} T(\psi) \sin(\psi) \cos(\psi) \mathop{d\psi},\\
        T_2 &= \frac{1}{2} \int\displaylimits_{-\pi/2}^{\pi/2} T(\psi) \frac{1}{2}(2\sin^2(\psi) -1) \cos(\psi) \mathop{d\psi}.
    \end{split}
\end{equation}

The $T_0$ temperature target translates to global mean surface temperature (GMST), the $T_1$ is interpreted as the inter-hemispheric temperature gradient and $T_2$ is interpreted as the equator-to-pole temperature gradient. From the model output these temperature targets can be evaluated using Eq. \ref{eq:Tpsi}. 


\textcolor{teal}{Een figure van het voorgeschreven aerosolveld is handig, maar of dat hier of in de introductie moet weet ik niet goed, ik neig naar hier. Een uitgebreidere bespreking van de datasests is denk ik ook gepast, maar of dit beperkt moet blijven tot 'van wie zijn ze en waar zijn ze te vinden' of dat het uitgebreider moet weet ik ook niet goed...}
\newpage

\section{Results Part I}
This section contains the results of the comparison between the CAM and WACCM simulations. 

\subsection{Temperature Targets}
The deviations from the temperature targets in Eq. \ref{eq:Tpsi} were calculated from 2-meter temperature for all simulations. The results are shown in Figure \ref{fig:Tgrad1}. The results from the SSP5-8.5 simulations with CAM and WACCM are near identical for $T_0$, showing comparable variability and arriving at virtually the same final GMST. The results from the graudal SAI simulations are also comparable, both simulations succeeding in maintaining 2020 GMST levels with comparable variability. 

Deviations from $T_1$ and $T_2$ are again comparable in CAM and WACCM. The SSP5-8.5 simulation in CAM arrives at a deviation from $T_1$ about 25\% lower than in WACCM, but given the large variability present in both models this is not significant. The gradual SAI simulation in CAM is again comparable to WACCM, maintaining $T_1$ and $T_2$ with similar variability. 

\begin{figure}[H]
	\centering
	\includegraphics[width=0.95\linewidth]{images/Tgrad_v.png}
	\caption{Deviations from temperature targets $T_0$, $T_1$, $T_2$ as compared to 2016-2025 mean, for the SSP5-8.5 and gradual SAI scenarios in CAM and WACCM.}
	\label{fig:Tgrad1}
\end{figure}


\subsection{2-meter Temperature and Precipitation in CAM}
Figure \ref{fig:CAM_scens} shows the Reference 2-meter temperature and precipitation, their anomalies for Control compared to the Reference and their anomalies for SAI 2020 compared to Control. In Figure \ref{fig:CAM_scens}(b) the Control period shows expected warming patterns, with the land area warming more than the sea and the poles warming more relatively due to polar amplification. Higher warming off the coast of Antarctica and in the Arctic Ocean indicates a retreat of sea-ice at both poles. The anomalous warming over the eastern Pacific ocean resembles the spatial sea surface temperature pattern of El Ni\~no. The slight cooling over the northern Atlantic ocean resembles a North Atlantic warming hole, generally attributed to changes in ocean heat transport and weakening of the Atlantic meridional overturning circulation (AMOC) but also atmospheric circulation changes due to anthropogenic forcing \parencite{menary2018anatomy,he2022}. 

The SAI 2020 anomaly compared to the Control, shown in Figure \ref{fig:CAM_scens}(c), shows that SAI is able to compensate for most of the warming in Control. The land area is cooled more than the sea, as are the poles. Sea-ice retreat is largely prevented in both the Arctic and the Antarctic. The anomalous warming over the eastern Pacific ocean is not prevented, as well as the North Atlantic warming hole. 

The precipitation anomaly in Control, shown in Figure \ref{fig:CAM_scens}(e), shows a general increase in precipitation. Most significant drying can be attributed to a shift in the intertropical convergence zone (ITCZ), with a southward shift over the Atlantic and eastern Pacific Ocean, expected patterns under increased anthropogenic forcing \parencite{mamalakis2021zonally}. The shift over the eastern Pacific Ocean is possibly related to the El Ni\~no pattern observed in Fig. \ref{fig:CAM_scens}(b). No significant shift or other change in ITCZ is observed over the Indian and western Pacific Ocean.

In Figure \ref{fig:CAM_scens}(f) the precipitation anomaly of SAI 2020 compared to Control shows opposite trends, apparently preventing most increase in precipitation. The southward shift of the ITCZ is mostly prevented, though not entirely over the eastern Pacific.

\begin{figure}[H]
	\centering
	\includegraphics[width=0.95\linewidth]{images/CAM_scens.png}
	\caption{CAM model results; (a): Reference annual mean 2-meter temperature. (b): Annual mean 2-meter temperature anomaly for Control compared to Reference, Reference shown in black contours in 10°C intervals (c): Annual mean 2-meter temperature anomaly for SAI 2020 compared to Control, contours as in (b) but for Control. (d): Reference annual mean precipitation in mm/day, 4 mm/day shown in blue contours, $<0.3$mm/day shown in red contours with hatching. (e) Annual mean precipitation anomaly for Control compared to Reference, contours as in (d). (f): Annual mean precipitation anomaly for SAI 2020 compared to Control, contours as in (d) but for Control.}
	\label{fig:CAM_scens}
\end{figure}


\subsection{SAI 2020 2-meter Temperature Anomalies in CAM and WACCM}
The annual and seasonal mean 2-meter temperature anomalies of SAI 2020 compared to the Reference for CAM and WACCM are shown in Figure \ref{fig:TREFHT_20ref}. Figure \ref{fig:TREFHT_20ref}(c) shows the difference between (a) and (b), i.e. the inter-model difference corrected for their difference in SAI 2020. Figures \ref{fig:TREFHT_20ref}(f) and (i) show the annual and seasonal zonal mean anomalies for CAM and WACCM respectively. 

The annual mean anomalies in Figures \ref{fig:TREFHT_20ref}(a) and (b) show a generally similar pattern of warming over the tropics, cooling over the subtropics and warming over the poles. The Arctic warms more in CAM than it does in WACCM, at most 3°C. Both models show significant cooling over the nothern Atlantic Ocean, again resembling the North Atlantic warming hole seen in Figure \ref{fig:CAM_scens}, CAM more so than WACCM. The El Ni\~no pattern in the eastern Atlantic is visible in both models, though more pronounced in CAM. 

Figure \ref{fig:TREFHT_20ref}(c) confirms the patterns described above. The figure also highlights the difference in response over North America and off the coast of Antarctica between 0° and 30°E, where CAM slightly cools and WACCM slightly warms, leading to a significant difference.

In Figures \ref{fig:TREFHT_20ref}(d) and (e), the seasonal response of CAM is shown. Comparison between the two seasons and the annual mean shows that most warming at the poles occurs during their respective winter. This is further confirmed by the zonal mean anomaly in Figure \ref{fig:TREFHT_20ref}(f). The El Ni\~no pattern is stronger in magnitude in JJA, though more expansive in DJF, though this is not visible in the zonal mean anomaly. 

The same patterns are visible in Figures \ref{fig:TREFHT_20ref}(a) and (b), with the Antarctic as a whole warming in the Southern Hemisphere winter and the Barentsz sea in the Arctic warming significantly in the Northern Hemisphere winter. This pattern is confirmed by Figure \ref{fig:TREFHT_20ref}. 

\begin{figure}[H]
	\centering
	\includegraphics[width=0.95\linewidth]{images/TREFHT_20ref.png}
	\caption{(a,b): Annual mean 2-meter temperature anomalies of SAI 2020 compared to Reference in (a) CAM and (b) WACCM. Reference mean temperature shown in black contours in 10°C intervals. (c): Difference between CAM and WACCM temperature anomalies. WACCM anomaly shown in black contours in 1°C intervals. (d,e): CAM anomalies as in (a,b) for (d) JJA and (e) DJF. (f): CAM zonal average temperature anomaly as in (a,b), annual mean anomaly in black, JJA anomaly in red, DJF anomaly in blue. (g,h,i): analogous to (d,e,f) for WACCM.}
	\label{fig:TREFHT_20ref}
\end{figure}


\subsection{SAI 2020 Precipitation Anomalies in CAM and WACCM}
The annual and seasonal mean precipitation anomalies of SAI 2020 compared to Reference for CAM and WACCM are shown in Figure \ref{fig:PRECT_20ref}. Figure \ref{fig:PRECT_20ref}(c) shows the difference between (a) and (b) as above in Figure \ref{fig:TREFHT_20ref}, note that this difference is calculated in percentage points. Figures \ref{fig:PRECT_20ref}(f) and (i) show the annual and seasonal zonal mean anomalies for CAM and WACCM respectively. 

The annual mean anomalies in Figures \ref{fig:PRECT_20ref} and \ref{fig:PRECT_20ref} show a similar pattern overall, with a small decrease in precipitation over most of the globe and larger increase in dry areas and the tropics. CAM shows a much larger precipitation increase over the eastern Pacific Ocean, with a southward shift of the ITCZ, while WACCM shows a larger increase over dry areas like the Arabian Peninsula. 

The same patterns are observed in the seasonal precipitation anomalies in Figures \ref{fig:PRECT_20ref}(d), (e), (g) and (h), with the exception of southern Brazil, where CAM shows a larger increase than WACCM in the JJA mean. The zonal mean anomalies in Figures \ref{Figure}(f) and (i) also show that precipitation in the ITCZ increases twice as much in CAM compared to WACCM. The increase in the Arctic is also about twice as high in CAM as it is in WACCM, while the Antarctic shows the opposite. 

The observerd precipitation trends correlate with the observed 2-meter temperature trends, in both the annual and seasonal results. A higher 2-meter temperature increase correlates to a higher precipitation increase, most notably in the eastern Pacific Ocean and the Arctic and Antarctic. 

\begin{figure}[H]
	\centering
	\includegraphics[width=0.95\linewidth]{images/PRECT_20ref.png}
	\caption{(a,b): Annual mean precipitation anomalies of SAI 2020 compared to Reference in (a) CAM and (b) WACCM. Reference mean precipitation shown, 4 mm/day in blue, $<0.3$mm/day in red with hatching. (c): Difference between CAM and WACCM precipitation anomalies. WACCM Control mean precipitation as in (b). (d,e): CAM anomalies as in (a,b) for (d) JJA and (e) DJF. (f): CAM zonal average precipitation anomaly as in (a,b), annual mean anomaly in black, JJA anomaly in red, DJF anomaly in blue. (g,h,i): analogous to (d,e,f) for WACCM.}
	\label{fig:PRECT_20ref}
\end{figure}


\subsection{Absolute temperature and precipitation differences between CAM and WACCM}
The absolute temperature differences for Reference, Control and SAI 2020 are shown in \ref{fig:CAM_WACCM}. The 2-meter temperature difference reveals that CAM is much warmer than CAM on the Northern Hemisphere, with the largest difference in the Arctic where the difference exceeds 7° in Reference and SAI 2020. In contrast, CAM is slightly cooler in the Southern Hemisphere, with the difference being generally less than 2°C except for a few areas in Reference.

This difference is lesser in Control, but still a clear difference between the two hemispheres is visible. The difference is worsened in SAI 2020, with the Arctic warming much more in CAM than in WACCM. 

As for precipitation, there is a small difference in Reference, with cam being slighly wetter over the tropics and dryer over the western Pacific Ocean. Differences shift in Control, with most notably a stark increase in precipitation over the eastern Pacific Ocean in CAM compared to WACCM. The same is seen in SAI 2020, with the ITCZ also shifting more clearly southward in CAM compared to WACCM. 

\begin{figure}[H]
	\centering
	\includegraphics[width=0.95\linewidth]{images/CAM_WACCM.png}
	\caption{(a-c): Reference, Control and SAI 2020 CAM-WACCM 2-meter temperature differences, WACCM temperature in black contours in 10°C intervals. (d-e): Analogous to (a-c) for precipitation, 4 mm/day in blue contours, $<0.3$ mm/day in red contours with hatching.}
	\label{fig:CAM_WACCM}
\end{figure}


\subsection{SAI 2020 Potential Temperature and Zonal Wind Anomalies in CAM and WACCM}
The zonal mean potential temperature $\theta$ anomaly and the zonal mean zonal wind anomaly for CAM and WACCM is shown in Figure \ref{fig:th_U_full}, for both the annual and seasonal mean. 
In Figures \ref{fig:th_U_full}(a)-(f) the potential temperature anomaly shows significant warming in the stratosphere, in a pattern reminiscent of the aerosol field from section \ref{emulator_pt1}. The warming extends to the poles only in the summer hemisphere, due to the poles only receiving solar radiation in their respective summer. The cooling observed in CAM over the Arctic in winter is not observed in WACCM at the same magnitude, any cooling extends to about the same altitude, but is not as severe as it is in CAM. CAM also shows more cooling than WACCM in the Antarctic winter, though the difference is not as stark as in the Arctic. The cooling effect of greenhouse gases in the upper stratosphere is visible in CAM and WACCM in the annual and seasonal means. There is no significant potential temperature change observed below about 300 hPa. 

The zonal mean zonal wind anomaly in Figures \ref{fig:th_U_full}(g)-(l) shows a large increase in zonal winds in the upper stratosphere in both hemispheres. The seasonal mean figures reveal that most zonal wind increase occurs in the winter hemisphere. The polar night jet (PNJ) forms in the upper stratosphere in winter, driven by the large latitudinal temperature gradient during the polar night. The peak increase in the winter hemisphere is observed above the area with the largest increase in the latitudinal temperature gradient, see Figures \ref{fig:th_U_full}(b), (c), (e) and (f), discussed above. The largest anomaly is visible on the equator side of the PNJ, indicating an increase in strenght and an equatorward shift. This pattern is observed everywhere, except in the Arctic winter in WACCM (Fig. \ref{fig:th_U_full}(l)). The PNJ only increases in strength slightly, correlating to the temperature anomaly (Fig. \ref{fig:th_U_full}(f)). 

Lower statospheric and tropospheric winds are largely unchanged in both CAM and WACCM, with only significant changes reaching the surface in the Arctic winter in CAM. 

\begin{figure}[H]
	\centering
	\includegraphics[width=0.95\linewidth]{images/th_U_full.png}
	\caption{(a-f): Zonal mean potential temperature anomaly for (a-c) CAM and (d-f) WACCM, annual, JJA and DJF mean shown for both models. Reference potential temperature shown in black contours, zonal mean zonal wind anomalies shown in magenta contours. (g-l): Zonal mean zonal wind anomaly for WACCM and CAM analogous to (a-f), Reference zonal wind shown in black contours.}
	\label{fig:th_U_full}
\end{figure}
\newpage

\section{Discussion Part I}
\begin{itemize}
    \item the $T_2$ temperature is limited as a metric when temperatures are not uniform, in CAM the warming hole over the Northern Atlantic leads to warming over the Arctic, zonal mean temperatures show large variation in temperature over the Northern Hemisphere, but $T_2$ is still stable.
    \item comparing precipitation between two models that exhibit vastly different behaviour for precipitation is unreliable. CAM is wetter than WACCM in ITCZ region, but dryer in other outside of tropics. Even more so in Control and SAI 2020, with CAM showing a much larger increase in the tropics in both scenarios.  
    \item coarse horizontal resolution leads to not so reliable results for precipitation (Ren\'e zei dit, moet ik nog verder naar kijken)
    \item thick layers in upper stratosphere in WACCM (CMIP6) make validation harder
    \item over equator upper stratosphere differences between models remain unexplained
    \item all of the above is exacerbated by the limited data that comes with using only one run for each scenario in each model
\end{itemize}
\newpage

\part{Southern Hemisphere Atmospheric Circulation}
\section{Methods Part II}
\subsection{Rapid Cooling Experiment}
The rapid cooling experiment is branched from the SSP5-8.5 simulation in 2080, SAI is then deployed to restore temperatures to 1.5°C above pre-industrial levels. The control algorithm is adjusted for the first up to six years to prevent extremely high aerosol concentrations that would result in too rapid cooling. 

Shown in Figure [T0T1T2tot2130] are the temperature targets from \ref{eq:Tpsi} for the SSP5-5.8, gradual SAI and rapid cooling SAI simulations. After a few years of SAI the T$_0$ target is reached and maintained, like in the gradual SAI simulation. The T$_1$ target stabilised after ??? years, showing ??? behaviour. The T$_2$ target shows ??? behaviour (un)like in the gradual SAI simulation.

[INSERT T0T1T2tot2130 FIGURE HERE]

\textcolor{teal}{hier moet ik het figuur nog voor maken, gewoon nog geen zin in gehad. Is een plaatje van de spatial distribution ook relevant, zoals in pt1 ook voor SAI 2020 is gemaakt?}


\subsection{Definition of Scenarios and Time Periods for Part II}
The two simulations from Part I are referred to in the same way in this second part, namely the SSP5-8.5 simulation and the gradual SAI simulation. The simulation with the rapid cooling experiment is referred to as the rapid cooling SAI simulation. 

All simulations considered in this second part were extended from 2100 to 2130. This extension provides further insight in the long-term effects of deploying SAI. Especially in the rapid cooling SAI scenario the extension provides time for the climate system to adjust to the `shock' it experienced from SAI. 

Throughout this second part, four time periods are used to visualise and interpret the results from the simulations. As in part I, for each period the 20-year mean is taken, unless specified otherwise. These periods are defined as follows:

\begin{itemize}
    \item \textbf{Reference} The period 2016-2035 of the SSP5-8.5 simulation.
    \item \textbf{Control} The period 2111-2130 of the SSP5-8.5 simulation.
    \item \textbf{SAI 2020} The period 2111-2130 of the gradual SAI simulation.
    \item \textbf{SAI 2080} The period 2111-2130 of the rapid cooling SAI simulation.
\end{itemize} 
\textcolor{teal}{misschien beter in een tabel}

\subsection{Thermal Wind}
To calculate thermal winds from the temperature gradients, the thermal wind balance equation is used. As the vertical layers of the model are converted to pressure coordinates, the equation takes the form 

\begin{equation}
    \frac{\partial v_g}{\partial p} = - \frac{R}{pf_0} \frac{\partial T}{\partial x};\ 
    \frac{\partial u_g}{\partial p} = \frac{R}{pf_0} \frac{\partial T}{\partial y},
\end{equation}

where $v_g$,$u_g$ is the geostrophic wind in meridional and zonal directions respectively, $R = 286.9$ J kg$^{1}$ K$^{1}$ is the specific gas constant, $p$ is pressure, $f_0 = 2\Omega \sin \varphi$ is the coriolis parameter at the chosen reference latitude $\varphi$, $\frac{\partial T}{\partial x,y}$ is the layer-mean temperature gradient in zonal and meridional direction respectively. Rewriting and integrating gives us

\begin{equation}
    \begin{split}
        \int\displaylimits_{p_0}^{p_1} \partial v_g &= \int\displaylimits_{p_0}^{p_1} - \frac{R}{f_0} \frac{\partial T}{\partial x} \partial \ln p; \\
        \int\displaylimits_{p_0}^{p_1} \partial u_g &= \int\displaylimits_{p_0}^{p_1} \frac{R}{f_0} \frac{\partial T}{\partial y} \partial \ln p,
    \end{split}
\end{equation}

where $p_{0,1}$ are the lower and upper boundaries of the model layer, respectively, so that $p_1 < p_0$. Because $T$ is the layer-mean temperature and $R$ and $f_0$ are constants, we can evaluate this integral to find the thermal wind in the layer between $p_0$ and $p_1$

\begin{equation}
    \begin{split}
        v_T = v_g(p_1) - v_g(p_0) &= \frac{R}{f_0} \frac{\partial T}{\partial x} \ln\left(\frac{p_0}{p_1}\right);\\
        u_T = u_g(p_1) - u_g(p_0) &= - \frac{R}{f_0} \frac{\partial T}{\partial y} \ln\left(\frac{p_0}{p_1}\right).
    \end{split}
\end{equation}

To find the thermal wind at a given model layer, we take the cumulative sum of the model layers below it plus the meridional or zonal wind of the model layer below the lowest layer. We do this because the thermal wind in the lower model layers will most likely not produce stable results. To increase stability in our calculations, we assume that the thermal wind below the start of the integration is is equal to the model wind. This gives us for the thermal wind at model layer $i$ 

\begin{equation}
    \begin{split}
        v_{T,i}(p_i) &= v_{p_{-1}} + \frac{R}{f_0} \sum_{i=0}^{i} \frac{\partial T_i}{\partial x} \ln \left(\frac{p_i}{p_{i+1}}\right),\\
        u_{T,i}(p_i) &= u_{p_{-1}} + \frac{R}{f_0} \sum_{i=0}^{i} - \frac{\partial T_i}{\partial y} \ln \left(\frac{p_i}{p_{i+1}}\right).
    \end{split}
\end{equation}

\subsection{Kinetic Energy and Eddy Kinetic Energy}
The CAM model woks with wind in the form of $u = \overline{u} + u^{\ast}$, $v = \overline{v} + v^{\ast}$, with $u,v$ the total wind, $\overline{u},\overline{v}$ the time-mean wind and $u^\ast,v^\ast$ the deviation from the mean wind. The model output contains monthly averages $\overline{u},\overline{v}$ and $\overline{u^2},\overline{v^2}$. 

The kinetic energy $KE$ can thus be found from the model results directly through

\begin{equation}
    KE = \frac{1}{2} \left( \overline{u^2} + \overline{v^2} \right).
\end{equation}

The eddy kinetic energy $EKE$ can be found through 

\begin{equation}
    EKE = \frac{1}{2} \left( \overline{u^{\ast 2}} + \overline{v^{\ast 2}} \right),
\end{equation}

\noindent with $\overline{u^{\ast 2}}, \overline{v^{\ast 2}}$ found through
\begin{equation}
    \begin{split}
        \overline{u^2} &= \overline{\left( \overline{u} + u^\ast \right) \left( \overline{u} + u^\ast \right)},\\
        &= \overline{\overline{u}^2 + 2 \overline{u}u^\ast + u^{\ast 2}},\\
        &= \overline{u}^2 + \overline{u^{\ast 2}} \Rightarrow \overline{u^{\ast 2}} = \overline{u^2} - \overline{u}^2, 
    \end{split}
\end{equation}

which can be found with the model output. 

\subsection{Jet Intensity Maps}
The jet intensity maps are made by counting the number of times the model value of interest passes a set threshold. Each timestep and coordinate is evaluated individually, after which the set is summed over time and the vertical dimension. This result is then normalised to the number of timesteps multiplied by the number of vertical model levels included in the analysis. This results in a fraction that represents the time and vertical extent of the atmospheric jet. 

To visualise the Polar night jet (PNJ), in the upper stratosphere, and the subtropical jet (STJ), in the lower stratosphere, this method is applied to the zonal wind fields. For the eddy-driven jet (EDJ), also in the lower stratosphere, this method is applied to the eddy kinetic energy fields. 


\subsection{Climograph}
\textcolor{teal}{Is uitleg hiervan nodig?}
\newpage

\section{Results Part II}
\subsection{Potential Temperature and Zonal Wind Anomaly}
The zonal-mean potential temperature and potential temperature anomalies for Control, SAI 2020 and SAI 2080 are shown in Figure \ref{fig:th_U_zmdiff_full}, for the annual, JJA and DJF mean. The black countours indicate the zonal-mean zonal wind in m/s. 

In Control we see the potential temperature anomaly signature of increased greenhouse gases, warming at the surface and the troposphere, extending to the lower stratosphere and cooling in the upper stratosphere. The cooling trend in the upper stratosphere is present in SAI 2020 and SAI 2080 as well, but over the Antarctic SAI 2020 and SAI 2080 cool significantly more. With SAI the potential temperature increases where SAI is deployed in the lower stratosphere, conecentrated between 15°S and 30°S around 50 hPa. The warming extends over the Antarctic only in the summer (DJF), because of the lack of incoming solar radiation during the polar night in winter (JJA). The stratosphere in SAI 2020 is about 2.5K warmer than in SAI 2080. The tropospheric temperatures are successfully held at present day values in either SAI scenario, within a margin of $\pm$2.5K.

In JJA (winter) we see that the temperature anomaly in SAI 2020 and SAI 2080 is opposite of the trend in Control. The Antarctic stratosphere cools with SAI instead of warming as it does under global warming. In DJF (summer) the anomaly is much more comparable, with warming over the Antarctic in all scenarios.

The zonal winds are mostly driven by the meridional temperature gradient through the thermal wind balance. Qualitatively, we see in Reference that the subtropical jet (STJ) in the lower stratosphere forms above a steep meridional temperature gradient around 30°S.

We also see an increase in wind speed around 50°S where we would expect to see the eddy-driven jet (EDJ), also above a relatively steep meridional gradient. From the wind speeds alone it seems the EDJ is weaker than the STJ, however, we know that the EDJ is generally stronger. This is because of the waviness of the EDJ, so clear identification must follow from eddy-activity, covered in Section \ref{EDJ_sec}. 

Because the STJ and EDJ and their anomalies are strongest in winter, we will focus on this season in any further analysis of these jets in the lower stratosphere.

In the upper stratosphere we see in JJA a strong westerly wind, the polar night jet (PNJ). Comparison of the winter and summer meridional temperature gradient reveals that this jet is driven by temperature gradients as well. The meridional temperature gradient is positive (equator-to-pole) in JJA, resulting in strong westerlies. In DJF the gradient flattens in the stratosphere – even turning slightly negative in the upper stratosphere - preventing strong westerly winds from forming. 

In all scenarios the PNJ shifts equatorward, in Control because the latitude of the highest temperature gradient shifts over the whole stratosphere, in SAI 2020 and SAI 2080 because the temperature gradient strongly increases on the equatorward side in the lower stratosphere. This will be discussed in more detail in Section \ref{upperstrat}.

\begin{figure}[H]
	\centering
	\includegraphics[width=\linewidth]{images/th_U_zmdiff_full.png}
	\caption{Annual and seasonal mean zonal-mean potential temperature (shading and magenta contours) and zonal-mean zonal wind (black contours, m/s) for Reference (first row) and Control, SAI 2020 and SAI 2080 anomaly compared to Reference (second to fourth row).}
	\label{fig:th_U_zmdiff_full}
\end{figure}

\subsection{Results Part II: Lower Stratosphere}\label{lowerstrat}

\subsubsection{Subtropical Jet}
In Figure \ref{fig:th_U_zmdiff_full} the zonal mean of the zonal component of the integrated thermal wind is shown, together with the observed zonal-mean zonal wind in black countours. The observed wind and integrated thermal wind show a strikingly similar pattern, with the observed wind consistently lower than the integrated thermal wind due to friction effects, increasingly so further up in the atmosphere. 

The change in magnitude of the lower stratosphere wind strongly correlates with the change in magnitude of the integrated thermal wind. The STJ shifts upward and equatorwad in Control, also showing a significant increase of up to 12 m/s in the upper regions. This pattern is consistent with the change in potential temperature in Figure \ref{fig:th_U_zmdiff_full}. See also Figure \ref{fig:th_U_zm_JJA} in Appendix \ref{app_th_U_JJA}, the meridional gradient of the 350K isotherm is much steeper around 30°S in Control compared to Reference.

In contrast, in both SAI 2020 and SAI 2080 the STJ decreases in strength, up to 6 m/s in the upper regions around 100 hPa. This possibly also indicates an equatorward shift, but slightly downward instead of upward. 

\begin{figure}[H]
	\centering
	\includegraphics[width=0.95\linewidth]{images/UT_U_zmdiff_JJA.png}
	\caption{JJA mean zonal-mean zonal thermal wind (shading and red contours) and zonal-mean zonal wind (black contours, m/s) for (a): Reference; (b-d): Control, SAI 2020 and SAI 2080 anomaly compared to Reference.}
	\label{fig:UT_U_zmdiff_JJA}
\end{figure}

On the subtropical jet intensity map in Figure \ref{fig:STJ_map_JJA} the same trends as above are observed. In Control the jet intensifies and shifts equatorward. The largest increase is observed in the eastern Pacific Ocean, in Reference the jet is relatively weak in this region, but in Control the jet is the at its peak here. The area between 120°W and 60°E increases the most relatively, the opposite hemisphere shows little change.

In SAI 2020 and SAI 2080 the jet is much less intense, but the spatial distribution remains largely unchanged. The jet weakens slightly more in SAI 2080 than in SAI 2020, but the difference does not seem significant. 

The location, height and mean of the maximum of the STJ reported in Figure \ref{fig:STJ_maxloc_JJA} largely confirm the patterns observed above. In Control, the STJ maximum shifts equatorwad between 120°W and 60°E and remains around the position of Reference in the other hemisphere, correlating to the increase in intensity. The upward shift of the maximum is about 25 hPa, and the mean maximum wind speed increases 6.2 m/s, or 12.7\%. 

The maximum wind speed in SAI 2020 and SAI 2080 remains mostly at Reference levels, deviating at most 1.1 m/s, or 2\%, in SAI 2020 which is not significant. As the maximum wind speed does not change considerably, but the jet does appears much less intense, this means the jet has become more constrained without losing much strength at its maximum. The location of the maximum in SAI 2020 and SAI 2080 has not shifted much overall, only significantly around 60°E. Looking at the intensity in Figure \ref{fig:STJ_map_JJA} this seems to be the result of the wind speeds in the eddy-driven jet (around 50°S) reaching similar levels as the STJ jet. The EDJ effectively competes with the STJ for the maximum wind speed and `wins' a considerable amount of times.


\begin{figure}[H]
	\centering
	\includegraphics[width=0.95\linewidth]{images/STJ_map_JJA.png}
	\caption{JJA subtropical jet intensity map of zonal wind, values counted when the threshold of 40 m/s was passed between 400 and 100 hPa, for (a) Reference, (b) Control, (c) SAI 2020 and (d) SAI 2080. 0°E is oriented towards the top of the figure.}
	\label{fig:STJ_map_JJA}
\end{figure}

\begin{figure}[H]
	\centering
	\includegraphics[width=0.48\linewidth]{images/STJ_maxloc_JJA.png}
	\caption{JJA mean location of the maximum of the subtropical jet, with corresponding longitudinal mean height and maximum eddy kinetic energy for Reference, Control, SAI 2020 and SAI 2080. 0°E is oriented towards the top of the figure.}
	\label{fig:STJ_maxloc_JJA}
\end{figure}


\subsubsection{Eddy-driven Jet}\label{EDJ_sec}
The EDJ is already visible in Figure \ref{fig:STJ_map_JJA}. The zonal wind in the Eastern Hemisphere reveals another jet structure next to the STJ. Beacuse this jet is driven by eddy activity, the zonal wind alone will not reveal much on its behaviour. We use eddy kinetic energy (EKE) as a measure for eddy activity. The zonal-mean EKE and EKE anomalies are shown in Figure \ref{fig:EKE_U_zmdiff_JJA}.

Around 50°S and 300 hPa, there is a peak in EKE, i.e. high eddy activity, this is where we identify the EDJ to be. In Control the EKE decreases on the lower equatorward side of the jet and increases on the upper poleward side, indicating that the jet shifts poleward and upward. It is not clear from this figure if the jet becomes more or less wavy. In SAI 2020 and SAI 2080 the EKE shows small changes compared to what is observed in Control. A slight decrease is observed around 45°S, in the area between the EDJ and the STJ, and stays close to Referene levels everywhere else. SAI 2020 responds more strongly than SAI 2080.

The eddy-driven jet intensity map in Figure \ref{fig:EDJ_map_JJA} also indicates an increase in intensity in Control and a poleward shift. The change in intensity in SAI 2020 and SAI 2080 is again much smaller, both weakening slightly while the overall spatial pattern remains largely unchanged. 

The location of the maximum in Figure \ref{fig:EDJ_maxloc_JJA} reveals that the EDJ remains largely in position in both SAI scenarios, but shifts poleward everywhere in Control. The upward shift identified in Figure \ref{fig:EKE_U_zmdiff_JJA} in Control is confirmed to be about 23 hPa. In SAI 2020 and SAI 2080 the jet maximum shifts slightly downward, though not significantly. In each scenario the mean maximum EKE decreases slightly but insignificantly, 2-4\%, again not providing much insight into any changes waviness of the jet. 

\begin{figure}[H]
	\centering
	\includegraphics[width=0.95\linewidth]{images/EKE_U_zmdiff_JJA.png}
	\caption{JJA mean zonal-mean eddy kinetic energy (shading) and zonal-mean zonal wind (contours, m/s) for (a): Reference; (b-d): Control, SAI 2020 and SAI 2080 anomaly compared to Reference.}
	\label{fig:EKE_U_zmdiff_JJA}
\end{figure}

\begin{figure}[H]
	\centering
	\includegraphics[width=0.95\linewidth]{images/EDJ_map_JJA.png}
	\caption{JJA eddy-driven jet intensity map of EKE, values counted when the threshold of 105 J/m$^3$ was passed between 600 and 150 hPa, for (a) Reference, (b) Control, (c) SAI 2020 and (d) SAI 2080. 0°E is oriented towards the top of the figure.}
	\label{fig:EDJ_map_JJA}
\end{figure}

\begin{figure}[H]
	\centering
	\includegraphics[width=0.48\linewidth]{images/EDJ_maxloc_JJA.png}
	\caption{JJA mean location of the maximum of the eddy-driven jet, with corresponding longitudinal mean height and maximum eddy kinetic energy for Reference, Control, SAI 2020 and SAI 2080. 0°E is oriented towards the top of the figure.}
	\label{fig:EDJ_maxloc_JJA}
\end{figure}
\newpage
The STJ has more kinetic energy than the EDJ, mostly because of the higher mean wind speeds. The EDJ has a proportionally larger eddy component and thus higher EKE. Comparing the changes in KE and EKE and zonal wind speed
will provide insight into the waviness of the two jets. 

In Figure \ref{fig:KE_U_zmdiff_JJA} the zonal-mean kinetic energy and zonal-mean zonal wind are shown. In Control the increase in KE follows the zonal wind patterns in sign and in magnitude. The KE increases with the increasing zonal wind accordingly in the upper regions of the STJ and decreases proportionally below. The KE also increases proportionally to wind speed in the EDJ. This indicates that the STJ and the EDJ do not become siginificantly more or less wavy under global warming.

For SAI 2020 and SAI 2080 the KE anomaly follows the zonal wind anomaly proportionally as well. Comparison with Figure \ref{fig:EKE_U_zmdiff_JJA} shows that the decrease in EKE for the EDJ is correlated with the decrease in zonal wind. The decrease in KE in the STJ shows that the constraining of the jet observed in Figure \ref{fig:STJ_map_JJA} leads to a loss of strength, even while maintaining maximum wind speed.


\begin{figure}[H]
	\centering
	\includegraphics[width=0.95\linewidth]{images/KE_U_zmdiff_JJA.png}
	\caption{JJA mean zonal-mean kinetic energy (shading) and zonal-mean zonal wind (contours, m/s) for (a): Reference; (b-d): Control, SAI 2020 and SAI 2080 anomaly compared to Reference.}
	\label{fig:KE_U_zmdiff_JJA}
\end{figure}


\newpage
\subsection{Results Part II: Upper Stratosphere}\label{upperstrat}
Because the polar night jet is strongest in the late winter to early spring, we will consider the August-September-October (ASO) mean in our analysis of the PNJ. Sudden stratospheric warming events will also be discussed in this section. 

\subsubsection{Polar Night Jet}
As was discussed above, in all scenarios the PNJ shifts equatorward, in Control because the latitude of the highest temperature gradient shifts, in SAI 2020 and SAI 2080 because the temperature gradient in the lower stratosphere strongly increases. Figure \ref{fig:PNJ_UT_U_zmdiff} shows the zonal component of the integrated thermal wind and the anomalies in Control, SAI 2020 and SAI 2080 in ASO. As in section \ref{lowerstrat}, the observed wind and the integrated thermal wind show the same patterns, with the integrated thermal wind consistently higher than the observed wind due to friction effects. 

In Figure \ref{fig:PNJ_UT_U_zmdiff} we see that the PNJ indeed shifts equatorward in Control, confirming that the changes in the PNJ are governed by thermal changes under global warming. Its strength does not significantly changes either way. In SAI 2020 and SAI 2080 the PNJ shifts equatorward as well, also governed by thermal changes from SAI, but in contrast to Control the wind speed increases significantly. The pattern in SAI 2080 is the same as in SAI 2020, but the increase is at least 4 m/s less.

\begin{figure}[H]
	\centering
	\includegraphics[width=0.95\linewidth]{images/PNJ_UT_U_zmdiff.png}
	\caption{ASO mean zonal-mean integrated thermal wind (shading and red contours) and zonal wind (contours, m/s) for (a): Reference; (b-d): Control, SAI 2020 and SAI 2080 anomaly compared to Reference.}
	\label{fig:PNJ_UT_U_zmdiff}
\end{figure}

The changes in kinetic energy per unit mass shown in Figure \ref{fig:PNJ_KE_U_zmdiff} correspond well with the changes in integrated thermal wind in Figure \ref{fig:PNJ_UT_U_zmdiff}. The eddy kinetic energy in Figure \ref{fig:PNJ_EKE_U_zmdiff} shows a different picture. In Control there is a stark decrease in EKE on the poleward side of the PNJ. This decrease is very large relative to the KE decrease, indicating a deacrease in eddy activity.

The decrease in EKE over the Antarctic is visible in SAI 2020 and SAI 2080 as well, though not as intensely. It still indicates a decrease in eddy activity. In the interior of the PNJ, where the KE increase is largest, the EKE shows a more broad increase. Most strikingly, in both SAI scenarios the EKE increases a little bit less where the zonal wind is strongest. Instead, the EKE increases the most on the flanks of the jet, indicating the jet becomes more broad and wavy. The magnitude by which this happens seems to correspond to the wind speed, as again SAI 2080 is slightly weaker than SAI 2020.


\begin{figure}[H]
	\centering
	\includegraphics[width=0.95\linewidth]{images/PNJ_KE_U_zmdiff.png}
	\caption{JJA mean zonal-mean kinetic energy (shading) and zonal wind (contours, m/s) for (a): Reference; (b-d): Control, SAI 2020 and SAI 2080 anomaly compared to Reference.}
	\label{fig:PNJ_KE_U_zmdiff}
\end{figure}


\begin{figure}[H]
	\centering
	\includegraphics[width=0.95\linewidth]{images/PNJ_EKE_U_zmdiff.png}
	\caption{JJA mean zonal-mean eddy kinetic energy (shading) and zonal wind (contours, m/s) for (a): Reference; (b-d): Control, SAI 2020 and SAI 2080 anomaly compared to Reference.}
	\label{fig:PNJ_EKE_U_zmdiff}
\end{figure}

The intensity map of the PNJ is shown in Figure \ref{fig:PNJ_map}. The PNJ intensifies in all scenarios, most strongly in SAI 2020, followed by SAI 2080 and lastly Control. This increase in control was not apparend from the zonal mean figures, possibly because the increase is not zonally uniform. The PNJ indeed grows more broad in SAI 2020 and SAI 2080, as observed above. The equatorward shift in all scenarios is more clearly observed in Figure \ref{fig:PNJ_maxloc}, where the mean location of the maximum observed wind speed is shown. The scenario with the largest shift varies per location, with hardly any shift in any scenario in the 0°-60°E section, Control shifting the most in the 120°-180°E section, and SAI 2020 shifting the most in the 240°-300°E section. The shift in SAI 2080 is mostly consistent with SAI 2020, but deviates in the 240°-300°E section, coinciding with Control there instead. Maximum wind speed remains at Reference levels in Control, but increases by 8.0 m/s, or 8.3\%, in SAI 2020 and 6.2 m/s, or 6.4\%, in SAI 2080.

\begin{figure}[H]
	\centering
	\includegraphics[width=0.95\linewidth]{images/PNJ_map.png}
	\caption{Polar night jet intensity map of zonal wind, values counted when the threshold of 80 m/s was passed at 10 hPa, for (a) Reference, (b) Control, (c) SAI 2020 and (d) SAI 2080. 0°E is oriented towards the top of the figure.}
	\label{fig:PNJ_map}
\end{figure}


\begin{figure}[H]
	\centering
	\includegraphics[width=0.48\linewidth]{images/PNJ_maxloc_latlon.png}
	\caption{Mean location of maximum wind speed at 10 hPa, with longitudinal mean maximum wind speed, for Reference, Control, SAI 2020 and SAI 2080. 0°E is oriented towards the top of the figure.}
	\label{fig:PNJ_maxloc}
\end{figure}

\subsubsection{Sudden Stratospheric Warming Events}
The decrease in EKE over the Antarctic observed in \ref{fig:PNJ_EKE_U_zmdiff} suggests a decrease in the occurence of sudden stratospheric warming events (SSW). Figure \ref{fig:PNJ_climographTU} shows the area weighted mean temperature of the 10 hPa level above 60°S, together with the zonal wind at 10 hPa and 60°S. In Reference there is a number of years where the temperature increases, and a number of years the zonal wind decreases compared to the mean. Though not explicitly identified, this happens in pairs, when the temperature rises, the winds decrease. In Control the frequency of these SSW-like conditions decreases significantly, the temperature and zonal wind bands narrow compared to Reference. In SAI 2020 there appears to be one year with SSW-like conditions, but again the temperature and zonal wind bands narrow. SAI 2080 shows the same trend. 

In all scenarios the zonal wind increases, as was already discussed in the results above, and the temperature decrease due to increased greenhouse gases is also visible. In all scenarios the PNJ forms earlier in the year, surpassing 80 m/s in early to mid-July in Control and in mid-June in SAI 2020 and SAI 2080, as opposed to mid-July in Reference. The PNJ also dissolves later in the year, decreasing to 80 m/s in late October in Control and early November in SAI 2020 and SAI 2080, as opposed to mid-October in Reference.

\begin{figure}[H]
	\centering
	\includegraphics[width=0.95\linewidth]{images/PNJ_climographTU.png}
	\caption{Climograph of area-weighted mean temperature of 60°-90°S at 10 hPa (black) and zonal-mean zonal wind at 60°S and 10 hPa (red) for (a) 2016-2045 and (b) 2101-2130 in the SSP5-8.5 experiment, (c) 2101-2130 in the gradual SAI experiment and (d) 2101-2130 in the rapid cooling SAI experiment.}
	\label{fig:PNJ_climographTU}
\end{figure}
\newpage 

\section{Discussion Part II}
\begin{itemize}
    \item Increase of STJ over eastern Pacific could be signal from increased GHG's seen in Control.
    \item Changes in STJ and EDJ are significant, but pale in comparison to changes observed in Control. Lower stratosphere response is generally opposite that of Control, general weakening where there is great strengthening in Control.
    \item PNJ increases greatly, equatorward shift in all scenarios, possibly the great shift in SAI is a compounding of SAI and Control effect. Increase is only observed in SAI scenarios, about 8\% and 6\% increase of mean maximum wind speed.
    \item SSW's seem to decrease in all scenarios, mostly in Control but there is not really a significant difference between SAIs and Control. The earlier establishment in SAI is possibly a compound effect of SAI and increased GHG's as timing also shifts in Control, though less pronounced.
    \item SSW-like state of the stratosphere indicates lower frequency of SSW's, but it is only the SSW-like state based on monthly data. SSW's in end-beginning of two months will likely not show up in monthly data anyway, even more so if they are minor SSW's. 30 years of one run each is too few datapoints to make any definitive statements of how SSW's in CESM respond to SAI (and global warming) in the Southern Hemisphere. 
    \item Monthly data limits insights into SSW's, limitations with eddy-activity as well.
    \item Any interannual variability is not examined, only 20-year climatology.
    \item Coarse horizontal resulotuion limits evaluation of (sub)mesoscale atmospheric circulation patterns.
    \item Application of emulator to SAI 2080 scenario flawed, cooling from a different state than scenario emulator was built off leads to over- and undercooling.
    \item As in part I, using one run for each scenario limits interpretation of all results, not just SSW's.
\end{itemize}

\newpage

\section{Conclusions and Outlook}
\subsection{Part I: Model Validation}

\noindent \textit{When applying the stratospheric aerosol emulator to a gradual SAI scenario in CESM2(CAM6), are we able to reproduce from \textcite{tilmes2020}
    \begin{enumerate}[label=\roman*]
        \item the temperature targets $T_0$, $T_1$ and $T_2$?
        \item the spatial and seasonal variations of temperature, precipitation and general aspects of the atmospheric circulation?
    \end{enumerate}
}


\subsection{Part II: Southern Hemisphere Atmospheric Circulation}

\noindent \textit{What are the impacts of the gradual SAI scenario on the Southern Hemisphere
    \begin{enumerate}[label=\roman*]
        \item subtropical and polar jets in the lower stratosphere? \\
                The subtropical jet is weaker and interferes with the polar jet/meanders less. Slight equatorward shift, but smaller and downward as opposed to upward in control. Localised increase in eastern pacific as in control. Eddy activity in the polar jet decreases on the equatorward side, region where polar and subtropical jet `meet'.
        \item polar night jet and sudden stratospheric warming events in the upper stratosphere?
    \end{enumerate}
How do the results for i and ii change under the rapid cooling SAI scenario?
}

\newpage

\section*{Acknowledgements}
First of all I would like to thank my supervisors Claudia Wieners, Michiel Baatsen and Jasper de Jong for their guidance, wisdom and understanding, and the weekly meetings that were as fun as they were productive. In particular Claudia for sharing her expertise on geoengineering and guiding me through the process of doing research, Michiel for sharing his seemingly endless knowledge on the atmosphere, and Jasper for helping me with the numerous technical and practical issues I encountered along the way.

I would also like to thank Daniel Pfl\"uger for the use of his dataset and helping met get started with my analysis of it, and for the comradery among activists at IMAU. 

I want to thank Willem for his support and care, helping me get through these past months. And lastly thanks to my friends and family for their support and encouragement and my fellow students for the warm atmosphere in the IMAU Master's room.
\newpage

\printbibliography
\addcontentsline{toc}{section}{References}

\section*{Code and Data Availability}
The CESM2(wACCM6) SSP5-8.5 1.5 data is available at \href{https://doi.org/10.26024/t49k-1016}{https://doi.org/10.26024/t49k-1016} from \textcite{tilmes2020}.\\
The CESM2(WACCM6) SSP5-8.5 data was contributed to CMIP6 and is available at \href{https://esgf-data.dkrz.de/search/cmip6-dkrz/}{https://esgf-data.dkrz.de/search/cmip6-dkrz/}.\\
The CESM2(CAM6) data for all simulations can be made available upon reasonable request as stated in \textcite{pfluger2024}.\\
All code that was used for the analysis and figures can be found at\\
\href{https://github.com/slingbeek/MScThesis/tree/main/SAIthesis\_analysis}{https://github.com/slingbeek/MScThesis/tree/main/SAIthesis\_analysis}.
\newpage
\appendix
\section{Absolute model differences between CAM and WACCM}\label{app_modeldiff}
The absolute temperature differences between CAM and WACCM for Reference, Control and SAI 2020 are shown in Figure \ref{fig:CAM_WACCM}. The 2-meter temperature difference reveals that CAM is much warmer than WACCM on the Northern Hemisphere, with the largest difference in the Arctic where the difference exceeds 7° in Reference and SAI 2020. In contrast, CAM is slightly cooler in the Southern Hemisphere, with the difference being generally less than 2°C except for a few areas in Reference.

This difference is lesser in Control, but still a clear difference between the two hemispheres is visible. The difference is worsened in SAI 2020, with the Arctic warming much more in CAM than in WACCM. 

As for precipitation, there is a small difference in Reference, with cam being slighly wetter over the tropics and dryer over the western Pacific Ocean. Differences shift in Control, with most notably a stark increase in precipitation over the eastern Pacific Ocean in CAM compared to WACCM. The same is seen in SAI 2020, with the ITCZ also shifting more clearly southward in CAM compared to WACCM. 

\begin{figure}[H]
	\centering
	\includegraphics[width=0.95\linewidth]{images/CAM_WACCM.png}
	\caption{(a-c): Reference, Control and SAI 2020 CAM-WACCM 2-meter temperature differences, WACCM temperature in black contours in 10°C intervals; (d-e): Analogous to (a-c) for precipitation, 4 mm/day in blue contours, $<0.3$ mm/day in red contours with hatching.}
	\label{fig:CAM_WACCM}
\end{figure}

\section{Potential Temperature and Zonal Wind JJA}\label{app_th_U_JJA}

\begin{figure}[H]
	\centering
	\includegraphics[width=0.95\linewidth]{images/th_U_zm_JJA.png}
	\caption{JJA mean zonal-mean potential temperature (shading and magenta contours) and zonal-mean zonal wind (black contours, m/s) for (a) Reference, (b) Control, (c) SAI 2020 and (d) SAI 2080.}
	\label{fig:th_U_zm_JJA}
\end{figure}

\section{2-meter Temperature in CAM}\label{app_trefht}
\begin{figure}[H]
	\centering
	\includegraphics[width=0.95\linewidth]{images/TREFHT_scens_ann.png}
	\caption{(a): 2-meter tempereature in Reference, contours in intervals of 10°C; (b-d): 2-meter temperature anomaly of (b) Control, (c) SAI 2020 and (d) SAI 2080 compared to Reference, contours as in (a).}
	\label{fig:TREFHT_scens_ann}
\end{figure}


\section{DJF eddy kinetic energy and Eddy Driven Jet Position}\label{app_EDJ}
\begin{figure}[H]
	\centering
	\includegraphics[width=0.95\linewidth]{images/EKE_U_zmdiff_DJF.png}
	\caption{DJF mean zonal-mean eddy kinetic energy (shading) and zonal-mean zonal wind (contours, m/s) for (a): Reference; (b-d): Control, SAI 2020 and SAI 2080 anomaly compared to Reference.}
	\label{fig:EKE_U_zmdiff_DJF}
\end{figure}

\begin{figure}[H]
	\centering
	\includegraphics[width=0.95\linewidth]{images/EDJ_maxloc.png}
	\caption{Mean location of the maximum of the subtropical jet, with corresponding longitudinal mean height and maximum eddy kinetic energy for Reference, Control, SAI 2020 and SAI 2080 for the (a) annual mean, (b) JJA mean and (c) DJF mean. 0°E is oriented towards the top of the figure.}
	\label{fig:EDJ_maxloc}
\end{figure}

\end{document}
