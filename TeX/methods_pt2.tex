\subsection{Rapid Cooling Experiment}
The rapid cooling experiment is branched from the SSP5-8.5 simulation in 2080, SAI is then deployed to restore temperatures to 1.5°C above pre-industrial levels. The control algorithm is adjusted for the first up to six years to prevent extremely high aerosol concentrations that would result in too rapid cooling. 

Shown in Figure [T0T1T2tot2130] are the temperature targets from \ref{eq:Tpsi} for the SSP5-5.8, gradual SAI and rapid cooling SAI simulations. After a few years of SAI the T$_0$ target is reached and maintained, like in the gradual SAI simulation. The T$_1$ target stabilised after ??? years, showing ??? behaviour. The T$_2$ target shows ??? behaviour (un)like in the gradual SAI simulation.

[INSERT T0T1T2tot2130 FIGURE HERE]

\textcolor{teal}{hier moet ik het figuur nog voor maken, gewoon nog geen zin in gehad. Is een plaatje van de spatial distribution ook relevant, zoals in pt1 ook voor SAI 2020 is gemaakt?}


\subsection{Definition of Scenarios and Time Periods for Part II}
The two simulations from Part I are referred to in the same way in this second part, namely the SSP5-8.5 simulation and the gradual SAI simulation. The simulation with the rapid cooling experiment is referred to as the rapid cooling SAI simulation. 

All simulations considered in this second part were extended from 2100 to 2130. This extension provides further insight in the long-term effects of deploying SAI. Especially in the rapid cooling SAI scenario the extension provides time for the climate system to adjust to the 'shock' it experienced from SAI. 

Throughout this second part, four time periods are used to visualise and interpret the results from the simulations. As in part I, for each period the 20-year mean is taken, unless specified otherwise. These periods are defined as follows:

\begin{itemize}
    \item \textbf{Reference} The period 2016-2035 of the SSP5-8.5 simulation.
    \item \textbf{Control} The period 2110-2129 of the SSP5-8.5 simulation.
    \item \textbf{SAI 2020} The period 2110-2129 of the gradual SAI simulation.
    \item \textbf{SAI 2080} The period 2110-2129 of the rapid cooling SAI simulation.
\end{itemize} 
\textcolor{teal}{misschien beter in een tabel}


\subsection{Jet Intensity Maps}
The jet intensity maps are made by counting the number of times the model value of interest passes a set threshold. Each timestep and coordinate is evaluated individually, after which the set is summed over time and the vertical dimension. This result is then normalised to the number of timesteps multiplied by the number of vertical model levels included in the analysis. This results in a fraction that represents the time and vertical extent of the atmospheric jet. 

To visualise the Polar night jet (PNJ), in the upper stratosphere, and the subtropical jet (STJ), in the lower stratosphere, this method is applied to zonal wind speeds. For the eddy-driven jet (EDJ), also in the lower stratosphere, this method is applied to the eddy kinetic energy. 

\subsection{Climograph}
\textcolor{teal}{Is uitleg hiervan nodig?}