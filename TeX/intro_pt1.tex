\subsection{Geoengineering in the Form of Stratospheric Aerosol Injections}
Geoengineering can be seen as a toolbox of methods that change the earth's climate system to achieve a desired effect. Limiting global warming to 1.5-2°C is the primary goal of geoengineering. The methods of geoengineering can be divided into two basic categories, Carbon Dioxide Removal (CDR) and Solar Radiation Management (SRM) \parencite{shepherd2009}. CDR focuses on lowering the amount of greenhouse gases in the atmosphere, by capturing CO$_2$ directly or enhancing and facilitating natural processes to speed up the extraction of CO$_2$ from the atmosphere or oceans. SRM on the other hand focuses on altering the earth's radiation budget. This can be done by increasing the amount of long wave radiation the earth emits into space. The most widely discussed approach is increasing how much short wave radiation from the sun is reflected back into space, i.e. increasing the planetary albedo. This is done for instance by making deserts more reflective or making clouds brighter. 

This thesis deals with SRM in the form of stratospheric aerosol injections (SAI). Aerosols or their precursors, in this case SO$_2$, are injected into the lower stratosphere where they reflect some of the incoming short wave radiation from the sun. This lowers the temperature of the earth's surface at the latitudes it was injected. As atmospheric circulation carries the aerosols around the globe zonally very quickly, only the latitude and season of the injection need to be considered. This choice was found to be crucial when determining the effectiveness of SAI on the poles by \textcite{duffey2023}. A more recent development in this regard started with \textcite{kravitz2017} and \textcite{macmartin2017}, and then the Geoengineering Large Ensemble (GLENS) project. 

\subsection{The GLENS Project and Subsequent CESM2 Simulations}
The GLENS project is a 20-member ensemble of gradual SAI simulations (Tilmes et al. 2018). From 2020 onwards SO$_2$ is injected at four injection points at $\pm$ 15°N and $\pm$ 30°N about 5 km above the tropopause. A feedback-control algorithm is used to adjust the injection amounts at each point individually based on departures from the temperature goals defined by \textcite{kravitz2016}. These are the global mean surface temperature, the inter-hemispheric temperature gradient and the equator-to-pole temperature gradient. This algorithm also tunes the injection amount to the seasonal response of the model. 

The GLENS project was performed using the Community Earth System Model version 1 (CESM1) \parencite{hurrell2013}, with the Whole Atmosphere Community Climate Model (WACCM) as its atmosphere component. This model uses a 0.9° latitude $\times$ 1.25° longitude rectangular grid with 70 vertical layers that reach up to 140 km, or about 10$^{-6}$ hPa. It includes comprehensive atmospheric chemistry for the middle atmosphere, incorporating ozone chemistry and chemistry relating to stratospheric sulfate formation. A simpler chemistry scheme is used for the troposphere. Aerosol chemistry is coupled to WACCM through the three-mode version of the Modal Aerosol Module (MAM3). \textcolor{teal}{past dit misschien beter in de methods, en dan voor cesm2?}

After the GLENS project the method using four injection points and a feedback algorithm was further explored using the more recent Community Earth System Model version 2, also using a newer version of the WACCM, now version 6 (CESM2(WACCM6)) \parencite{tilmes2020}. WACCM6 has the same horizontal and vertical resolution, but now includes comprehensive chemistry from the troposphere up to the lower thermosphere. The MAM4 modal aerosol scheme is used for the troposphere and stratosphere. \textcolor{teal}{dus dit ook naar methods? }

Because CESM2(WACCM) is so comprehensive in both vertical resolution and atmospheric chemistry, it is extremely well-suited to simulate SAI scenarios. However, due to this it is also a very cumbersome model, requiring a lot of computing time an resources. This limits its use in ensemble studies and studies considering a large number of scenarios. 

\subsection{CESM2 with CAM6 as an Emulator}
To reduce the amount of resources needed, \textcite{pfluger2024} introduce a method that uses a less comprehensive atmospheric component. The same CESM2 model configuration is used as in \textcite{tilmes2020}, but the Community Atmosphere Model version 6 (CAM6) is used for the atmosphere component \parencite{danabasoglu2020}. The aerosol field established in the CESM2(WACCM6) simulation is used as an external forcing. A feedforward-feedback algorithm is used to adjust the aerosol field as a whole to keep the global mean surface temperature (GMST) at 1.5°C above pre-industrial levels. So, in contrast to the CESM2(WACCM6) simulations, the temperature gradients are not adjusted for. 

To assess how well this method works as an emulator, the model results for surface temperature, precipitation, potential temperature and zonal wind are compared between CESM2(WACCM6) and CESM2(CAM6). \textcolor{teal}{dit mag uitgebreider}

To repeat, we formulate the following research questions:

\begin{enumerate}
    \item Does the CESM2(CAM6) succeed in maintaining global surface temperature through SAI, including spatial patterns?
    \item How does the CESM2(CAM6) SAI simulation perform compared to the CESM2(WACCM6) SAI simulation it was based on?
\end{enumerate}

The three temperature targets are calculated for all models. The model results for reference height temperature (or 2-meter temperature), will provide additional insight into the performance of CESM2(CAM6) in regards to regulating surface temperatures. Both annual and seasonal patterns are discussed. Additionally, precipitation patterns are discussed. To assess any differences in the vertical profile between the two models, the zonally averaged potential temperature and zonal winds are compared. 