For the second part of this thesis we consider the two CESM2(CAM) simulations to assess the effect of SAI on the high-latitude Southern Hemisphere atmospheric dynamics. Additionally, we introduce a third simulation that employs SAI from 2080 onward to rapidly cool the earth to 1.5°C above pre-industrial levels. This scenario is used to gain insight into the effects of rapid cooling of the climate system and how this differs from gradually increasing SAI to maintain GMST.

\subsection{Rapid Cooling with SAI as an Emeregency Intervention}
As current climate policies are insufficient to prevent global warming of 1.5°C or even 2°C (IPCC), the proactive gradual SAI scenario is unlikely to be implemented in time as well. Employing SAI much later on after prolonged heating of the climate system is a realistic, more reactive, scenario. The earth could be cooled very rapidly, allowing the end-of-century GMST goals to be reached after the climate system has endured an even longer period of warming than it has to date. The effects of such an intervention are largely uncertain. While it is rather certain that SAI could lower GMST, it is not certain what effects of previous warming can be reversed, if at all. 

Such a scenario is introduced in Pfl\"uger et al. (2024, pre-print). The SAI 2080 scenario is introduced, where SAI is employed from 2080 onwards to achieve rapid cooling to 1.5°C above pre-industrial levels. In this study the effects of SAI on ocean circulation is investigated, specifically how the rapid deployment of SAI compares to the gradual deployment of SAI. 

\subsection{The high-latitude Southern Hemisphere Atmosphere}
The high-latitude Southern Hemisphere is of particular interest in the context of global warming and the prevention of it. The Antarctic Ice Sheet could contribute greatly to global sea level rise if it were to become unstable under global warming. Observed instabilities of the West-Antarctic Ice Sheet (WAIS) alone could contribute to significant sea level rise (IPCC WG1). 

The Southern Hemisphere high stratosphere has low variability in the current climate, but any changes have far-reaching effects on the surface climate.  It has been observed that SH stratospheric polar vortex weakekening contributes to climate anomalies in Australia and New Zealand, southeast Africa and southern South America. Additionally, the wind stress over the ocean around Antarctica is weakened and the Ross and Amundsen seas experience warmer climate (Domeisen \& Butler, 2020). There is thus a clear interaction between the atmospheric dynamics and local climate in the Southern Hemisphere. 

\subsection{Previous Research}
The effect of SAI on the Antarctic ice sheet has been studied by McCusker et al. (2015), who found that a rapid introduction of sulphate aerosols in the stratosphere could not prevent the collapse of the WAIS. Sutter et al. (2013) found similar results. However, both studies use very simple aerosol schemes, that for instance only inject aerosols in the tropics. These types of schemes are known to lead to over-cooling of the tropics and under-cooling of the poles. As was laid out in Tilmes et al. (2020), the GLENS project aimed to implement an SAI scheme that could prevent the emergence of these patterns and was largely succesful in doing so. The CESM2(WACCM) repetition of this experiment and the
CAM emulator were also largely succesful in this regard.

Understanding the consequences of this type of SAI scheme on the Southern Hemisphere is crucial to be able to make informed decisions on the deployment of SAI in the future. 

\subsection{???}
In this second part the effect of SAI on the high-latitude Southern Hemisphere atmospheric dynamics is investigated, both in the gradual and rapid cooling scenarios. The CESM configuration used in the emulator does not include a dynamic ice model, so direct consequences to the AIS are not within the scope of this thesis. The focus here lies on atmospheric processes, as large scale circulation patterns largely dictate local climate in the Southern Hemisphere.

To repeat, we formulate the following research quesitons for the second part:

\begin{enumerate}
    \item How does SAI affect the Southern Hemisphere mid-level atmospheric circulation?
    \item How does SAI affect the Southern Hemisphere high latitude atmospheric circulation at high altitude (Polar Night Jet)?
    \item How does the rapid cooling SAI scenario differ from the gradual SAI scenario?
\end{enumerate}

\textcolor{teal}{het is nog niet een heel lopend verhaal voor mijn gevoel...}