\subsection{Part I: Model Validation}

\noindent \emph{When applying the stratospheric aerosol emulator to a gradual SAI scenario in CESM2(CAM6), are we able to reproduce from \textcite{tilmes2020}
\begin{enumerate}[label=\roman*]
    \item the temperature targets $T_0$, $T_1$ and $T_2$?\\
        \emph{The temperature targets are reached with similar end results and similar variability. Though the CAM response to the SSP5-8.5 scenario is significantly different from WACCM, the emulator with CAM is able to maintain 2020 levels for each target as well as WACCM is able to.}
    \item the spatial and seasonal variations of temperature, precipitation and general aspects of the atmospheric circulation?\\
        \emph{The surface temperature pattern observed in CAM varies from WACCM regionally, but globally corresponds well to WACCM. The differences in temperature are well within the range of inter-model variability. The precipitation patterns in CAM and WACCM agree overall, but the inter-model differences are larger than for the surface temperature. The difference in precipitation change mostly reflects model bias, with CAM favouring wetter tropics and a stronger southward shift of the ITCZ. \\
        The stratospheric warming from the injection of aerosols is nearly identical. The lack of ozone chemistry in CAM does appear to cause some additional cooling in the middle to upper stratosphere over the poles, but this does not cause concerning differences in the dynamical response of the model. The atmospheric dynamics of WACCM are replicated well by CAM, especially in the Southern Hemisphere.}
\end{enumerate}
}


\subsection{Part II: Southern Hemisphere Atmospheric Circulation}

\noindent \emph{What are the impacts of the gradual SAI scenario on the Southern Hemisphere
    \begin{enumerate}[label=\roman*]
        \item subtropical and eddy-driven jets in the lower stratosphere? \\
            \emph{The changes in the subtropical observed under the SSP5-8.5 scenario are prevented by SAI very well. The general and maximum wind speed are conserved, but the upper regions experience a small decrease, leading to an overall slight decrease in jet intensity and kinetic energy.
            The strong poleward shift of the EDJ is prevented almost completely by SAI. The EDJ does appear to become slightly less wavy, mostly on the equatorward side of the jet.
            The overshoot of SAI in both jets is small compared to the magnitude of change under global warming.}
        \item polar night jet and sudden stratospheric warming events in the upper stratosphere?\\
            \emph{The polar night jet is most affected by SAI compared to the STJ and the EDJ. The thermal response of the stratosphere leads to a strong equatorward shift and a considerable increase in strength. The equatorward and increasing trend observed under the SSP5-8.5 scenario is surpassed under SAI, with the jet growing more intense, broad and wavy. The PNJ grows more stable under SAI than present day climate and the frequency of sudden stratospheric warming events decreases significantly. Though the stability of the PNJ does not increase as much under SAI, SAI is not able to prevent this downward trend from emerging.}
    \end{enumerate}
How do the results for i and ii change under the rapid cooling SAI scenario?
\emph{For all trends observed in the atmospheric dynamics, the response from SAI 2080 is slightly weaker. This could be a result of the prolonged warming preceding the employment of SAI or slight differences in the exact employment of SAI due to the prolonged warming. The changes are very small and do not undermine the assertion that the rapid cooling scenario is objectively succesful and does not lead to vastly different behaviour in the atmospheric circulation in the Southern Hemisphere.}
}

\subsection{Outlook}
The emulator studied in this thesis is promising as a tool to explore the climate response to SAI scenarios while using comparatively limited computing resources. As with any climate model, a higher resolution experiment would provide better insight into precipitation changes. Higher resolution experiments also allow for analysis of sub-mesoscale phenomena like tropical cyclones. Analysis of daily model output would allow for a better assessment of sudden stratospheric warming events. Daily data also provides opportunities to study surface processes and weather, and possible teleconnections with the stratosphere. 
If computational resources allow it, expanding the temperature goals included in the feedback algorithm in the emulator could improve its performance. Adding a meridional component to the adjustment of the aerosol field could possibly open the door to more complex SAI scenarios that stray further from the SAI scenario the emulator was based on.

