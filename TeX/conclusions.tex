\subsection{Part I: Model Validation}

\noindent \textit{When applying the stratospheric aerosol emulator to a gradual SAI scenario in CESM2(CAM6), are we able to reproduce from \textcite{tilmes2020}
    \begin{enumerate}[label=\roman*]
        \item the temperature targets $T_0$, $T_1$ and $T_2$?
        \item the spatial and seasonal variations of temperature, precipitation and general aspects of the atmospheric circulation?
    \end{enumerate}
}


\subsection{Part II: Southern Hemisphere Atmospheric Circulation}

\noindent \textit{What are the impacts of the gradual SAI scenario on the Southern Hemisphere
    \begin{enumerate}[label=\roman*]
        \item subtropical and polar jets in the lower stratosphere? \\
                The subtropical jet is weaker and interferes with the polar jet/meanders less. Slight equatorward shift, but smaller and downward as opposed to upward in control. Localised increase in eastern pacific as in control. Eddy activity in the polar jet decreases on the equatorward side, region where polar and subtropical jet `meet'.
        \item polar night jet and sudden stratospheric warming events in the upper stratosphere?
    \end{enumerate}
How do the results for i and ii change under the rapid cooling SAI scenario?
}
