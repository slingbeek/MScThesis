As a model for SAI in a high-warming scenario, CAM performs objectively well. GMST is maintained at 2020 levels as expected and most effects of global warming are prevented in proportion; land and polar oceans warm more under high global warming and CAM is generally able to prevent this trend from emerging. The same can be said about precipitation trends, though not as decisively. The general wettening under high global warming is mostly prevented with SAI. 

In comparison to WACCM the SAI emulator with CAM is able to reproduce the behaviour of the temperatures $T_0, T_1, T_2$. Though the CAM response to the SSP5-8.5 scenario differs from WACCM, our analysis shows it is possible to maintain the temperature targets with the emulator with similar variability as WACCM. However, analysis of global surface temperature patterns shows that large differences in the response to SAI are still present, most prominently in the modelling of polar climate. This highlights the limitations to the use of the temperature targets as a metric by which to measure the success of an SAI strategy. If the goal is to create a uniformly effective SAI strategy with little to no regional bias, $T_0, T_1, T_2$ are not sufficient metrics.

Differences in model response are apparent for both temperature and precipitation, with the latter showing the most significant differences. However, we must emphasise that this difference is relatively small when compared to the absolute differences between the two models. In Appendix \ref{app_modeldiff} the absolute difference between CAM and WACCM is shown for temperature and precipitation. CAM is clearly biased towards a warmer Northern Hemisphere compared to CAM, up to 8°C in the Arctic. The 2°C difference observed in the Arctic is thus relatively small and in line with the bias already present in the models. The same is true for precipitation, with CAM having a bias towards wet tropics and a more southern ITCZ, with a positive difference of up to 1 mm/day. 

The potential temperature response to the introduction of aerosols in the stratosphere compares exceptionally well, in magnitude and location in annual, JJA and DJF data. The most notable differences emerge in the polar regions in the middle to upper stratosphere, most likely attributed to the lack of ozone chemistry in CAM. Nevertheless, the large scale atmospheric circulation patterns respond in a higlhy similar fashion. CAM is able to reproduce the patterns from WACCM, that have also been observed previous studies \parencite{bednarz2023climate}. The response is not uniform between the hemispheres though, especiallly the Polar Night Jet in the NH winter is markedly stronger in CAM compared to WACCM. With the exception of the high-latitude NH, the SAI emulator with CAM is able to reproduce the response of large scale atmospheric circulation to the deployment of SAI. We should also note that the data retrieved from CMIP6 for the Control simulation in WACCM only resolves three vertical layers between 30 and 3.5 hPa, less than half that of CAM, interfering with thorough analysis of this region. 

The aim for Part I of this thesis was to verify if the emulator built with the simpler CAM could generally replicate the SAI experiment performed with the chemically comprehensive WACCM. We looked into global, regional and seasonal trends to identify where the two models diverge in their response to SAI. However, this is a study that considers only one run for each experiment. The conclusions based on this little data are tentative and more runs are necessary to definitively determine the success of the SAI emulator with CAM. 
